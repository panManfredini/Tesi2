\chapter{The Search for neutral MSSM Higgs Bosons in the final state:
$\tau^{+}\tau^{-} \rightarrow e \mu + 4\nu$}

Discovering the mechanism responsible for electroweak
symmetry-breaking and the origin of mass for elementary particles has been
one of the major goals of the physics program at the Large Hadron
Collider~(LHC)~\cite{LHC}.  In the Standard Model (SM) this mechanism
requires the existence of a single scalar particle, the Higgs
boson~\cite{ENGLERT,HIGGS,HIGGS2,HIGGS3,Guralnik:1964eu}.
%The recent discovery of a particle with mass close to 125~GeV at the LHC with properties that resemble
%the ones of the Higgs boson~\cite{ATLASHiggsObservationJuly2012,CMSHiggsObservationJuly2012}
%is also compatible with several extension of the SM, and in particular with Super Symmetry (SUSY) scenarios~\cite{theo125,theo125_v2}. 
%When calculating radiative correction to the Higgs
%mass one encounters divergences, for the Higgs mass to be finite
%counter-term has to be added and a fine-tuning between the counter term and the divergences has to take place, this 
%is what is called the naturalness \cite{} problem. 
%Introduction of Super Symmetry (SUSY), a new symmetry that connects bosons and fermions, solve this problem. 
%the MSSM under the assumption that is a SM-like Higgs boson that can be 
%identified with one of  MSSM neutral Higgs bosons~
In the Minimal Supersymmetric extension of the Standard Model
(MSSM)~\cite{MSSM1, MSSM2} the Higgs sector is composed of two Higgs
doublets of opposite hyper-charge, resulting in five observable Higgs
bosons.  Two of these Higgs bosons are neutral and $CP$-even
($h$,$H$), one is neutral and $CP$-odd ($A$) and two are charged
($H^\pm$).  At tree level their properties such as masses, widths and
branching ratios can be predicted in terms of only two parameters,
often chosen to be the mass of the $CP$-odd Higgs boson $m_A$, and
the ratio of the vacuum expectation values of the two Higgs doublets
$\tan\beta$.  For relatively large values of $\tan\beta$ one of the
$CP$-even Higgs bosons is almost degenerate in mass with
$A$. Moreover, Higgs couplings to down (up) type fermions are enhanced
(suppressed) by $\tan\beta$, meaning that for large $\tan\beta$
bottom-quark and $\tau$ lepton will play a more important role than in
the SM case either for production and decay.

The production of the neutral $CP$-even MSSM Higgs bosons at hadron
colliders proceeds via the same processes as for the SM Higgs
production. However, the pseudoscalar $A$ instead cannot be produced
in association with gauge bosons or in vector boson fusion (VBF) at
tree-level, as this coupling is forbidden due to $CP$-invariance.  At
the LHC one of the most relevant production mechanisms for the MSSM
Higgs bosons is gluon-gluon fusion, $gg\rightarrow A/H/h$. In
addition, the production in association with $b$-quarks becomes
important for large value of $\tan\beta$ .  The decays of the neutral
MSSM Higgs bosons (in the assumption that all supersymmetric particle
are heavy enough) are the same as for the SM one with the already
cited exception of $A$, however the decay rates depend on a large
extent to the couplings with fermions and gauge bosons.

Searches for neutral MSSM Higgs bosons have been performed at
LEP~\cite{LEPLimits}, the
Tevatron~\cite{TevatronLimits1,TevatronLimits2,TevatronLimits3,TevatronLimits4,TevatronLimits5,TevatronLimits6}
and the LHC~\cite{CMSLimit, ATLASLimit}.  In this note a search for
neutral MSSM Higgs bosons with the ATLAS experiment at CERN is
presented, using proton-proton collisions at centre-of-mass energy of
8~TeV, with a recorded integrated luminosity of
$20.3 \ifb$.
The results of this search are interpreted in a model independent
fashion, as limits on the product of the cross section and branching
ratio for such a new particle, as well a limits on the MSSM in the
$m_{h}^{max}$ scenario \cite{MSSMmhmax}. Only b-quark associated and
gluon fusion are considered as production mode for the Higgs bosons, the search then focuses
on the subsequent decay into a
$\tau^+\tau^-$ pair. Furthermore, only cases in which both $\tau$
decays leptonically, with one decaying to an electron and the other to
a muon, are considered. This final state corresponds to a total
$\tau^+\tau^-$ branching ratio of approximately 6\%.
%However, due to the reduced QCD background, 
%this search can still compete with the hadron-hadron and the
%lepton-hadron cases. 
The analysis strategy is to
split the selected events in two categories by requesting the presence
(b-tag) or absence (b-veto) of a jet coming from a $b$-quark. This
solution helps to separate the contribution of the two production modes
and allows for optimisation of selections due to the 
different backgrounds for the different final states..

The signal topology is characterised by a final state with an
electron, a muon, and missing transverse energy due to the
presence of four neutrinos from the $\tau$ decays. Furthermore, the
final state may be split by the presence or
absence of a $b$-quark initiated jet, depending on the production
process.  The background processes which are considered in this study
are the production of $W$ and $Z$ bosons in association with
jets~($W/Z$+jets), pairs of top quarks~(\ttbar), single top
quark~(the so-called single-top) and pairs of electroweak gauge
bosons~($WW, WZ, ZZ$). Finally QCD multi-jet also forms a non-negligible
background due to its large production cross-section. Where possible
these backgrounds are estimated using data driven methods.

This note is structured as follows: The ATLAS detector is briefly
described in Section~\ref{sec:ATLAS}. In Section~\ref{sec:data_mc} the
collision data set, the Monte Carlo-simulated event samples as well as
hybrid (tau-embedded) data samples used in this study are
described. The reconstruction of physics objects, the trigger
requirements and the offline event selection are discussed in
Sections~\ref{sec:presel} and~\ref{sec:eventsel}. Background estimation methods are
described in Section~\ref{sec:BackgroundEstimation}, and the
systematic uncertainties are discussed in
Section~\ref{sec:Systematics}. A statistical analysis and resulting
exclusion limits are described in Section~\ref{sec:Results}, followed
by conclusions in Section~\ref{sec:Conclusions}.
Search of MMSM Higgs bla....
This chapter is divided in three sections, in the first one the motivation and the 
analysis strategy is presented, in section~\ref{} are discussed the problematics 
that arise when real data enters the game and ideas to estimate backround are needed, 
as well as systematics, in section~\ref{}  focus is set on results with a small introduction to
statistical methods.

%\section{Inside Neutral MSSM Higgs}
%\section{What is a Search?}
%\section{Search First Principle}
%\section{Search Elements}
%\section{The Search in a Nutshel}
\section{The Search Strategy}


\subsection{Motivation}

\subsection{What is a search?} 
Some statistical definitions Hypotesys 0, bla bla...

\subsection{Signal Topology}
Way to inhance signal that gives you sensitivity increase (b-tag, b-veto)
Higgs BR and production
\subsection{Selections}

\subsection{MMC}



