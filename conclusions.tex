\chapter{Summary and Conclusions}\label{chap:conclusion}
The  Higgs boson recently discovered  at the LHC 
with a mass of about $125$~GeV shows properties which are well compatible with the  predictions of
the Standard Model (SM). Nevertheless, this new particle can also
be accommodated within theories beyond the Standard Model. Among  them, 
supersymmetric extension of the SM are  theoretically favoured 
since they offer an elegant way to solve many open questions in the SM.
The minimal supersymmetric extension of the SM (MSSM) predicts the existence
of five Higgs bosons, two of them neutral and CP-even ($h$ and $H$), one neutral CP-odd ($A$) and two charged ($H^{\pm}$).
Given the large number of free parameter of the MSSM, 
benchmark scenarios have been introduced in which  most of the 
parameters are fixed leaving only two of them free which usually are chosen to be $m_A$ and $\tan\beta$,
the latter being the vacuum expectation value of the two MSSM Higgs doublets.
The $m_h^{mod}$ scenario where the observed boson is interpreted as 
the lightest MSSM CP-even Higgs boson $h$, while the other two MSSM Higgs bosons are  degenerate in mass
and decouple from gauge bosons, has acquired particular interest recently
in view of the Higgs boson discovery. 
A large part of the $m_A - \tan\beta$ plane is still currently         
unexplored  in this scenario, which is a strong motivation to pursue 
the search for additional neutral MSSM Higgs bosons.

%Two studies with the ATLAS detector at the LHC have been performed in this thesis: the search for the neutral MSSM Higgs bosons
%and an investigation on the prospect for enhancing the sensitivity of this search by using track-based jet
%in alternative to the canonical calorimeter-based jet reconstruction.

%%%%%%%%%%%%%%% ANALYSIS SUMMARY %%%%%%%%%%%%%%%%%%%
In this thesis a search for the neutral MSSM Higgs bosons has been performed using proton-proton collision data
recordered by the ATLAS experiment at the LHC in 2012 at a centre-of-mass energy
of 8~TeV which correspond to an integrated luminosity of 20.3$\,\text{fb}^{-1}$. The search focuses on  Higgs boson decays
into pairs of $\tau$ leptons which subsequently decay via  $\tau^+ \tau^- \rightarrow e \mu +4\nu$.
The signal production processes considered are gluon fusion and the production in association with b-quarks. To enhance the signal 
sensitivity,  the  events are split into two mutually exclusive categories based on the presence
or absence of b-tagged jets.
 
The main background contributions in this search are $\Ztautau$, $\ttbar$ and  diboson production and QCD multi-jet processes.
The contribution of the dominant $\Ztautau$ background is measured using a dedicated  signal-depleted control data sample
in order to reduce the systematic uncertainties of the simulation. The QCD multi-jet background contribution 
is also estimated from a dedicated data control sample since these processes are difficult to model. All  other 
background contributions are predicted by simulation. 
The background model is validated using several control data samples which are well described by the model.
Systematic uncertainties in the simulation of signal and background processes
from cross section calculations and the modelling of the detector response and in the 
background determined from  data are taken into account.

The data are interpreted in the MSSM $m_h^{mod}$ benchmark scenario as a function of $m_A$ and $ \tan\beta$
in the ranges $90 \leq m_A \leq 300$ GeV and $5 < \tan\beta < 60$, respectively.
The measured $\tau\tau$ invariant mass distributions is compared to the predictions of the  background-only and the
signal-plus-background hypotheses. No significant excess above the estimated Standard 
Model background has been found. Exclusion limits are derived in the $m_A - \tan\beta$ plane 
in the $m_h^{mod}$ benchmark scenario. Values of $\tan\beta \apprge 10$ are excluded 
in the mass range $90 < m_A < 200$~GeV. The highest local p-value for the background only hypothesis 
is observed in the mass range $250< m_A <300$~GeV which corresponds to a $1.9\,\sigma$ excess.

The results  are also interpreted in other MSSM benchmark scenarios and in a  less model-dependent 
way in terms of upper limits on the production cross section times branching fraction 
of a generic scalar boson $\phi$ with  mass  $m_\phi$ via the  processes 
$pp \rightarrow b\bar{b}\phi$ and $gg \rightarrow \phi$.
The result of this search are  combined with the ones from  searches in other $\tau\tau$ 
decay final states. The combined limits considerably constrain
the allowed parameter space and represent the current best upper limit at  high values of $m_A > 600$~GeV. 
%%%%%%%%%%%%%%%%%%%%%%%%%%%%%%%

%%%%%%%%%%%%%%%%%%%%%%%%%%%%% Trackjets %%%%%%%%%%%%%%%%%%%%%%%%%%%%%%%%%%%%%%
The search for the neutral MSSM Higgs bosons suffers  strongly from the 
poor b-tagging performance for  low-energy b-jet produced in association 
with the signal. This is due mainly to two effects:
the reconstruction and calibration of jets from energy deposits in the calorimeters are stronlgy 
deteriorated by pile-up effects of multiple proton interaction per bunch crossing and 
in particular for low jet transverse momentum, furthermore the intrisic b-tagging algorithms 
performances drops rapidly with the transverse momentuma of the jet due mainly to mis-association of 
tracks to the jet, secondary interactions and multiple scattering.
Improvements on the b-jet identification performance and jet reconstruction for low energies would result 
in a major improvement of the sensitivity of the Higgs boson search in association with b-quarks.
An alternative b-jet identification procedure in which the b-tagging algorithm is applied on 
track-based jets instead of the  canonical calorimeter jets has been studied.
The track-based jets consist of inner detector tracks, the high spatial track resolution
allows for the association of the jet to their point of origin making the track-jets 
considerably more robust against pile-up effects.

The performance of the b-tagging algorithms in track-based jets has been investigated for the first time in this thesis.
One of the major systematic uncertainties arises from the incoplete description of the inner detector (ID) material
 budget in simulation.
A novel technique for determining the systematic uncertainties in the track-jets energy scale and
reconstruction efficiency  due to the ID material budget mismodelling  has been developed. 
For track-jets with transverse momenta below 20~GeV, the  uncertainty  in the energy scale  due to 
material budget mismodelling is estimated to range from 2\% to 4\% depending on the track-jet 
momentum and number of  tracks in the jet.

It has been shown that for b-jets with transverse momenta below 20~GeV the track-based jet reconstruction provide a higher
jet reconstruction efficiency than the calorimeter based one and is, therefore, more suitable  for
low $\pt$ b-tagging. The sensitivity of the search for the neutral MSSM Higgs bosons
in associatiated production with b-quarks can be improved by
 up to a factor of two if track-jet are employed for b-tagging instead of  calorimeter-based jets.
However, to exploit the full power of this technique a dedicated calibration of the 
b-tagging algorithms is  needed for  track-based jets, furthermore systematic uncertainties
in track-jet energy scale due to the simulated description of the underlying event, parton showering
and fragmentation functions need to be evaluated.














