\chapter{Summary and Conclusions}
The recently discovered Higgs boson at the LHC 
with a mass of about $125$~GeV shows properties fully compatible with the one predicted from the Standard Model (SM).
Nevertheless, such a new particle can still be accommodated within theories beyond the
Standard Model. Among  them, supersymmetry is a theoretically favoured model 
since it offer an elegant way to solve several of the current theoretical limitations of the SM.
The minimal supersymmetric extension of the SM (MSSM) predicts the existence
of five Higgs bosons, two of them neutral and CP-even $h$ and $H$, one neutral CP-odd $A$ and two charged $H^{\pm}$.
Given the large number of free parameter of the MSSM, several benchmark scenarios have been introduced 
which  fix the value of these parameters leaving only two of them free to vary independently, usually chosen to be $m_A$ and $\tan\beta$.
A benchmark scenario of particular interest is the $m_h^{mod}$ where the observed Higgs boson is interpreted as 
the lightest MSSM CP-even Higgs boson $h$, while the other two MSSM Higgs bosons tend to be degenerate in mass
and decouple from gauge bosons. A relatively large part of the $\tan\beta -m_A$ plane in this scenario is still currently 
unexplored, which is a strong motivation to pursue the search for additional neutral MSSM Higgs bosons.

Two studies with the ATLAS detector at the LHC have been performed in this thesis: the search for the neutral MSSM Higgs bosons
and an investigation on the prospect for enhancing the sensitivity of this search by using track-based jet
in alternative to the canonical calorimeter-based jet reconstruction.

%%%%%%%%%%%%%%% ANALYSIS SUMMARY %%%%%%%%%%%%%%%%%%%
The search for the neutral MSSM Higgs bosons has been performed using LHC proton-proton collision data at a centre-of-mass energy
of 8~TeV corresponding to an integrated luminosity of 20.3$\,\text{fb}^{-1}$. The search focuses on the Higgs boson decays
into a pair of $\tau$ leptons which subsequently decays via  $\tau^+ \tau^- \rightarrow e \mu +4\nu$.
The signal production processes considered are gluon fusion and the production in association with b-quarks. To enhance the signal 
sensitivity,  the  selected events are categorised into two mutually exclusive categories based on the presence
and on the absence of b-tagged jets in the final state. 
The main background contribution for this search are $\Ztautau$, $\ttbar$, diboson and QCD multi-jet processes.
The contribution of the dominant $\Ztautau$ background process is measured in a dedicated  signal-depleted control data sample
in order to reduce the systematic uncertainties of the simulation. Similarly, the QCD multi-jet background contribution 
is also estimated from a dedicated data control sample since this background process is hard to model. All the other 
backgrounds contribution are predicted using simulation. 
The background model is validated in several control data sample and  shows a good description of the data.
Systematic uncertainties  on cross section calculations and the modelling of the detector response for 
simulated signal and background processes are taken into account. For background processes  determined from  data,
the uncertainties of the measurement methods are evaluated.
The search is performed within the MSSM $m_h^{mod}$ benchmark scenario
scanning the $m_A - \tan\beta$ plane in the range $90 \leq m_A \leq 300$ GeV and $5 < \tan\beta < 60$.
The statistical interpretation of the data is based on the 
comparison of the observed $\tau\tau$ invariant mass distributions with the predictions of the  background-only and signal-plus-background
hypotheses. No significant excess of event above the estimated Standard Model background has been found. 
Exclusion limits are derived in the $m_A - \tan\beta$ plane for the $m_h^{mod}$ benchmark scenario where values of $\tan\beta \apprge 10$ are excluded 
for the mass range $90 < m_A < 200$,
%something on limits
the highest local p-value
is observed in the mass rage $250< m_A <300$~GeV and corresponds to $1.9\,\sigma$.
The results  are also interpreted in other scenarios and in a  less model-dependent way in terms of exclusion  limit on the cross section for 
the production of a generic Higgs boson $\phi$ with  mass  $m_\phi$ via the  processes $pp \rightarrow b\bar{b}\phi$ and $gg \rightarrow \phi$.
The result of the search are  combined with the one from  searches in other $\tau\tau$ final states, the combined limits considerably constrain
the allowed parameter space and represent the current best upper limit for high $m_A$ values. 
%%%%%%%%%%%%%%%%%%%%%%%%%%%%%%%

%%%%%%%%%%%%%%%%%%%%%%%%%%%%% Trackjets %%%%%%%%%%%%%%%%%%%%%%%%%%%%%%%%%%%%%%
The search for the neutral MSSM Higgs bosons suffers  strongly from poor b-tagging performance due to the relatively low energy of
the b-jet produced in association with the signal. Improvements on the b-tagging performance would result in a major improvement of the search sensitivity
for the Higgs boson produced in association with b-quarks. 
An alternative b-jet identification procedure in which the b-tagging algorithm is applied on 
track-based jets (track-jets) instead that on canonical calorimeter jets has been studied in this thesis.
The calorimeter jets are reconstructed
from the energy clusters in the calorimeter, while the track-based jets consist of inner detector tracks,
the precise track information makes the latter considerably more robust to pile-up effects.
The performance of the b-tagging algorithms on track-based jets is investigated for the first time.
Systematic uncertainties on track-based jet reconstruction are also studied: one of the major systematic
uncertainty arises from the mismodelling in simulation of the inner detector material (ID)  budget.
A novel technique for addressing the ID material budget systematic uncertainty in track-jets energy scale and 
reconstruction efficiency has been developed. For low transverse momentum track-jets uncertainty due to 
material budget mismodelling in the energy scale is estimated to range from 2\% to 4\% depending on track-jet momentum and number of 
associated tracks.

It was shown that for jets with low transverse momentum the track-jets provide a higher
b-hadron reconstruction efficiency than calorimeter jets and are more suitable  for
low $\pt$ b-tagging. The sensitivity to the neutral MSSM Higgs boson produced in association with b-quarks can be improved 
 up to a factor two if track-jet reconstruction is employed instead of the calorimeter based one.
However, to exploit the full power of this technique a dedicated calibration of the 
b-tagging algorithms is  needed for  track-based jets. 














