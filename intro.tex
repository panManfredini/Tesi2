\chapter{Introduction}

 
The Standard Model (SM) of particle physics is a theory that describes at quantum level the interaction 
between matter and radiation, it is a very predictive  theoretical framework and 
has been largely confirmed by experiments.  The description of the interaction
between the elementary constituent of matter (spin 1/2 fermions) via the electromagnetic, weak and strong interaction
is mediated by gauge (spin 1) bosons and is based on the principle of local gauge invariance.
To introduce the mass for all these  particles without spoiling the electroweak symmetry of the theory
a mechanism of spontaneous symmetry breaking is necessary, this is achieved by introducing 
an additional complex scalar field, the Higgs field, which breaks the symmetry and
give rise, in addition to the particles masses, to a new scalar particle, the Higgs boson.

The recent discovery at the  Large Hadron Collider (LHC) of a new boson of mass
of about 125~GeV by the ATLAS and CMS experiments~\cite{AHiggsO,CHiggsO} is in agreement with the 
SM prediction for the Higgs boson.
The most recent measurements  of its properties~\cite{ASpin0,ACouplings,CFermions,CWidth} shows 
that they are fully compatible with the SM Higgs boson. However, it remains an open question
whether this new particle is the only missing piece of the electroweak symmetry breaking
sector or whether it is one of several Higgs bosons predicted in  theories 
that go beyond the SM.  Among  them, supersymmetry is a theoretically favoured model
since it offer an elegant way to solve several of the current theoretical limitations of the SM.
The minimal supersymmetric extension of the SM (MSSM) predicts the existence
of five Higgs bosons, two of them neutral and CP-even $h$ and $H$, one neutral CP-odd $A$ and two charged $H^{\pm}$.
This thesis present a search for the neutral MSSM Higgs bosons performed with proton-proton collision data recordered by the ATLAS
experiment at the LHC. Chapter~\ref{chap:theory} of this thesis 
is devoted to introduce the Minimal Supersymmetric extension of the Standard Model, focusing 
in particular on its Higgs sector, with a description of the expected neutral MSSM Higgs bosons
phenomenology.


The search presented in this thesis is based on 20.3 $\text{fb}^{-1}$ of proton-proton collision data at 
a centre-of-mass energy of $\sqrt{s} = 8$~TeV recordered by the ATLAS experiment.
The latter is one of the several experiments
at the LHC, taking data with a  multi-purpose detector designed  to search for  a wide range of new 
physics phenomena and to perform  precision measurements of known Standard Model processes.
The ATLAS detector consist of four sub-detectors which are installed  cylindrically around the
beam pipe, symmetrically in the forward and backward direction with respect to the proton beams.
The innermost sub-detector is the inner detector, followed by the electromagnetic calorimeter, the hadronic calorimeter and finally
a muon spectrometer in the outermost layer.
An overview of the ATLAS experiment and of the Large Hadron Collider complex is given in Chapter~\ref{chap:detector}. 

The data recordered by the ATLAS experiment need to undergo several steps of offline reconstruction 
before being ready for analysis, the physics object reconstruction and quality criteria used in this thesis are described in 
Chapter~\ref{chap:obj}.

The main topic of this thesis is the search for the neutral MSSM Higgs bosons decaying into pairs of tau leptons,
each subsequently decaying into an electron or muon and two neutrinos.
This final state correspond to 6\% of the total branching fraction of a di-tau leptons system decays. 
In spite the  limited branching fraction, this final state has a competitive sensitivity with the other channels, especially
for low Higgs boson mass.
The search is performed for the two most significant production modes, gluon fusion and in association with
b-quarks. The  selected events are categorised into two mutually exclusive categories based on the presence
and on the absence of b-tagged jets in the final state.
The search for the neutral MSSM Higgs boson is presented in Chapter~\ref{chap:anal}.

The Higgs bosons produced in association with b-quark is characterised by the presence of relatively low-transverse momentum
jets (originating from b-quarks), the reconstruction and calibration of calorimeter jets for such a low transverse momentum 
suffers strongly of pile-up effects. This is one of the main reason for sensitivity loss of this search in the category that 
requires a b-tagged jet. An alternative approach would be to use the so-called track-based jets reconstruction, 
which relies only on identified inner detector tracks for the jet reconstruction.  The measure of inner detector tracks
is very precise and is possible to associate them with the corresponding interaction vertex,  
this feature makes track-based jets more robust against pile-up effects than calorimeter-based jets. 
A study on the prospect for enhancing the sensitivity of the neutral MSSM 
Higgs boson search  by using track-based jet is presented in Chapter~\ref{chap:trackjet}.











