\chapter{Introduction}

 
The Standard Model (SM) of particle physics  describes the strong and electroweak interactions
of quarks and leptons and has been confirmed extremely well by experiments at energy scales below about 1 TeV. 
The interactions between the elementary constituents of matter are mediated by gauge bosons 
based on the principle of local gauge invariance.
Masses for all these  particles are introduced without spoiling the electroweak gauge symmetry 
via the mechanism of spontaneous symmetry breaking. An additional complex scalar field is required for this
purpose which give rise to a new scalar particle, the Higgs boson.

The recent discovery at the  Large Hadron Collider (LHC) of a new boson of mass
of about 125~GeV by the ATLAS and CMS experiments~\cite{AHiggsO,CHiggsO} is in agreement with the 
Higgs boson prediction by the SM. The measurements  of its properties~\cite{ASpin0,ACouplings,CFermions,CWidth}
are well compatible with those of the SM Higgs boson. However, the question remains 
whether this new particle is the only missing piece of the electroweak symmetry breaking
sector or whether it is one of several Higgs bosons as predicted by many models beyond the SM.
Supersymmetric extension of the SM are theoretically favoured since they
offer an elegant solution to limitations of the SM.
The minimal supersymmetric extension of the SM (MSSM) predicts the existence
of five Higgs bosons, two of them neutral and CP-even ($h$ and $H$), one neutral and CP-odd ($A$) and two charged ($H^{\pm}$).
In this thesis, a search for the neutral MSSM Higgs bosons is performed with 20.3$\,\text{fb}^{-1}$ of
proton-proton collision data at a centre-of-mass energy of 8~TeV recorded by the ATLAS experiment at the LHC. 
Chapter~\ref{chap:theory} is devoted to an  introduction to the MSSM focusing  on the Higgs sector
and on the neutral MSSM Higgs boson phenomenology.

An overview of the ATLAS experiment is given in Chapter~\ref{chap:detector}. 
The ATLAS detector consist of four main sub-detectors, the inner detector,
the electromagnetic and hadronic calorimeters and the muon spectrometer. These sub-detectors are installed 
cylindrically around the beam pipe in the central barrel part and in disks in the end-caps which are symmetrical in
forward and backward direction with respect to the proton beams.
The data recorded by the ATLAS experiment undergo several steps of offline reconstruction 
before being ready for analysis. The physics object reconstruction and data quality criteria used in this thesis are described in 
Chapter~\ref{chap:obj}.

In Chapter~\ref{chap:anal}, the search for the neutral MSSM Higgs bosons  performed 
in  $A/h/H \rightarrow \tau^+ \tau^- \rightarrow e\mu + 4\nu$ decays is discussed.
This final state corresponds to 6\% of the total decay rate of the two $\tau$ leptons. 
In spite of the  rather small branching fraction, this final state provides a signal 
sensitivity which is competitive with the other channels, especially
for low Higgs boson masses, because of the high background rejection.
The  events are split into two mutually exclusive categories based on the presence or
absence of b-tagged jets indicating the two main Higgs production modes,
in association with b-quarks and via gluon fusion, respectively.

The Higgs boson production in association with b-quarks is characterised by the presence of low 
transverse momentum b-jets.
The reconstruction and calibration of low transverse momenta jets from energy deposits in the calorimeters are strongly 
deteriorated by pile-up effects of multiple proton interactions per bunch crossing, causing a large loss of efficiency for the $A/h/H$ search
in the b-tagged category. As an alternative, jet reconstruction based on inner detector 
tracks has been studied for the purpose of b-tagging.
The inner detector tracks are associated to their original interaction vertex which 
makes track-based jet reconstruction more robust against pile-up effects than calorimeter-based jets. 
A study on the prospects for enhancing the sensitivity of the neutral MSSM 
Higgs boson search  by using track-based b-jet identification is presented in Chapter~\ref{chap:trackjet}.

A summary of the neutral MSSM Higgs boson search and of the prospects for its improvement by employing track-based jet reconstruction
is given in Chapter~\ref{chap:conclusion}.











