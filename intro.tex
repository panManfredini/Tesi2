\chapter{Introduction}

 
The Standard Model of particle physics is a theory that describes at quantum level the interacion 
between matter and radiation. It is a very predictive and succesfull theoretical framework
which has been largely confirmed by experimets.  The recent discovery of a new boson of mass
$\sim 126$~GeV by the ATLAS and CMS experiments~\cite{AHiggsO,CHiggsO} at Large Hadron Collider 
in agreement with the SM prediction is another success of this theory.
The most recent measurements \cite{ASpin0,ACouplings,CFermions,CWidth} of its
properties shows this new boson to be, within experimental uncertainties, fully
compatible with the SM Higgs boson. However, it remains an open question
whether this new particle is the only missing piece of the electroweak symmetry breaking
sector or whether it is one of several Higgs bosons predicted in  theories 
that go beyond the SM. Among all of them, Supersymmetry  is a theoretically favoured scenario
as the most predictive framework beyond the Standard Model. Chapter~\ref{chap:theory} of this thesis 
is devoed to introduce the Minimal Supersymmetric extension of the Standard Model, focusing 
in particular on its Higgs sector, with a description of the expected neutral MSSM Higgs bosons
phenomenology.

The search presented in this thesis is based on 20.3 $\text{fb}^{-1}$ of proton-proton collision data at 
a centre-of-mass energy of $\sqrt{s} = 8$~TeV recordered by the ATLAS experimet at the LHC during 2012.
An overview of the ATLAS experiment and of the Large Hadron Collider complex is given in Chapter~\ref{chap:detector}. 
The data recordered by the ATLAS experiment need to undergo several steps of offline reconstruction 
before being ready for analysis, the physics object reconstruction and quality criteria used are described in 
Chapter~\ref{chap:obj}.

The main topic of this thesis is the search for the neutral MSSM Higgs bosons decaying into pairs of tau leptons,
each subsequently decaying into an electron or muon and two neutrinos.
This final state correspond to 6\% of the total branching fraction of a di-tau leptons system decays, even limited by the BR
this final state has a competitive sensitivity with the other channels at low mass.
The search is performed for the two most significant production modes, gluon fusion and in association with
b-quarks. The search is performed in two complementary event catergory, either with or without reconstructed b-jet
and are optimized separately. Chapter~\ref{chap:anal}.

In general the Higgs bosons produced in association with b-quark is carachterized by the presence of relatively low-transverse momentum
jets (originating from b-quarks), the reconstruction and calibration of calorimeter jets suffers strongly for such a low transverse
momentum, which is one of the main reason for sensitivity loss in the category that requires a tagged jet.
An alternative approach would be to 

Chapter~\ref{chap:trackjet}.

