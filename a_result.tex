
\section{Results}

\subsection{LHC Procedure For Limits Setting}
\label{sec:limits}
Redefinition of marked poisson  with likelihood, comment on Histfactory

def of likelihood:
$$
\mathcal{L}(\text{data}|\mu, \boldsymbol{\theta}) = \text{Poisson(data} | \mu \cdot s(\boldsymbol{\theta}) + b(\boldsymbol{\theta})) 
	\cdot f(\boldsymbol{\theta} | \hat{\boldsymbol{\theta}})
$$

%and the $\text{Poisson(data} | \mu \cdot s(\boldsymbol{\theta})$ 
and the $\text{Poisson(data }| \mu \cdot s(\boldsymbol{\theta})) $
stands for a product of Poisson probabilities to observe  events in the bin \textit{i}
$$
\prod_{i} \frac{(\mu s_i +b_i)^{n_i}}{n_i!} ~ e^{-\mu s_i -b_i}
$$

To compiute the compatibility of the data with the $H_{0}$ and $H_{1}$ hypotesis, and then exclusion limits 
one needs to define a test statistic. The test statistic, which has already been defined in section~\ref{},
is a function of the data which returns a real value. One can in principle use any test statistic, however, 
given the size of the test (probability to reject the null hypotesis when is true) one would like to have 
a test statistic which has the highest power $1 - \beta$ possible (probability to reject the 
null hypotesys when it is false), this means that the test statistic should have 
different distribution for the two hypotesis under test. Figure~\ref{} shows an example
of the distribution of an hipotetical test statistic for two hypotesis.
It has been shown by Neuman-P~\cite{} that in case of simple hyptesis (probability model without any parameter),
then the test statistic with the highest power is the ratio of the likelihood calculated with the two hypotesis:
The standard procedure at the LHC is to use the following test statistic \cite{} based on the likelihood ratio:
$$
\tilde{q_{\mu}} ~ = ~ -2 \text{ln} ~ \frac{\mathcal{L}(\text{data}|\mu, \hat{\boldsymbol{\theta}}_{\mu})}{\mathcal{L}(\text{data}|\hat{\mu}, \hat{\boldsymbol{\theta}})}
\quad \text{with the constraint} \quad 0 \leq \hat{\mu} \leq \mu
$$
 where $\hat{\mu}$ and $\hat{\boldsymbol{\theta}}$ are the maximum likelihood estimators for $\mu$ and $\boldsymbol{\theta}$ given the data, 
whereas $\hat{\boldsymbol{\theta}}_{\mu}$ is the maximul likelihood estimator of $\boldsymbol{\theta}$ given the data but considering
a signal streght of value $\mu$. to be noted that $\tilde{q_{\mu}}$ is increasing with increasing disagreement between data and the $\mu$ hypotesis under test.
The procedure for limits setting follows five steps:
\begin{enumerate}
	\item The signal hypotesis with signal streght $\mu$ is assumed, under this assumpion a set of 
	\textit{pseudo-data} is generated for different values of $\mu$.

	\item  $\tilde{q_{\mu}}$ is calculated for each of the \textit{pseudo-dataset} and each signal hypotesis generating
	the expected probability density function for $\tilde{q_{\mu}}$ given $\mu$, 
	$\text{f}(\tilde{q_{\mu}} ~| ~ \mu, \hat{\boldsymbol{\theta}}_{\mu},H_1)$.

	\item One does the same thing foe the null hypotesis, generate pseudo-data with the distribution of background only and 
	generate  the $\text{f}(\tilde{q_{\mu}} ~ | ~ \mu = 0, \hat{\boldsymbol{\theta}}_{0}, H_0)$.

	\item Once one has the pdf for the signal and signal + background hypotesis one can define for a given dataset (that can be this time
	real data or again pseudodata)  two p-values, which are the probability to obtain data less compatible with the hypotesis in consideration:
	$$
	p_{s+b} = P(\tilde{q_{\mu}} > \tilde{q_{\mu}}^{observed} ~ | ~ H_1)  \quad \text{for any given value of} ~  \mu ~ \text{and}$$
	$$ 
	p_{b} = P(\tilde{q_{\mu=0}} > \tilde{q_{\mu=0}}^{observed} ~ | ~ H_0)
	$$
	Calculate the ratio of this two probability and get what is called the $CL_{s} = p_{s+b} / p_{b}$ \cite{}.

	\item If for a given $\mu$ is obtained $CL_{s} ~ \leq ~ \alpha $ one states that the signal hypotesis (with that $\mu$) 
	is excluded with (1 - $\alpha$) $CL_{s}$ confidence level. To get the 95\% confidence level upper limit on $\mu$,
	denoted as $\mu^{95}$ one adjust $\mu$ until $CL_{s} = 0.05$. 
\end{enumerate}
This is a quite complicated prescription, however it has simple explaination in terms of confidence intervals:

For each $\mu$ is possible to define $\tilde{q_{\mu}}^{95}$ for which the probability P$(\tilde{q_{\mu}} > \tilde{q_{\mu}}^{95} ~ | H_1$ ) = 5\%,
	this means that if $H_1$ is true, this hypotesis will be rejected in 5\% of the cases.

 For each pseudo-data sample one can calculate then $\mu^{95}$, which is the biggest value of $\mu$ which gives $\tilde{q_{\mu}} > \tilde{q_{\mu}}^{95}$

 Generate a large set of pseudo-data under the hypotesis $H_{0}$ and calculate  $\mu^{95}$ for each of them, by costruction one would have
	that if $H_1$ is true for $\mu > \mu^{95}$ one has $\tilde{q_{\mu}} > \tilde{q_{\mu}}^{95}$ only 5\% of the cases.
	
So one it is said that the $\mu > \mu^{95}$ are excluded by data under the hypotesis $H_1$ at 95\% of Confidence Level, meaning that 
for value of $\mu > \mu^{95}$  one would reject the $H_1$ when is true at most 5\% of the time. 

\subsection{Exclusion Limits}
The procedure described in section~\ref{sec:limits} is actually the one which was used for the SM Higgs, for the MSSM there is a further 
complication: the cross section and the masses of $h$ and $H$ depends on $m_A$ and $\tan \beta$, then one has to consider in the signal 
model three Higges for which in a particular scenario the masses and cross section are defined for a given point in the $\tan \beta - m_A$
plane, so the procedure described previously has to be repeated for each point in that plane. For a given scenario, 
a point in the  $\tan \beta - m_A$ plane is excluded with 95\% $CL_{s}$ confidence level if $\mu^{95} \leq 1$ for that point. 
Only a limited number of $\tan \beta - m_A$ points are gerated, a linear interpolation is used to determine the $\tan \beta$ excluded for a given
$m_A$.

Add some part of limits of the note.

 

\subsection{Summary}

