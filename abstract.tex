\thispagestyle{empty}
\newgeometry{left=3.7cm, right=3.7cm,bottom=4cm ,top=2cm}
\begin{abstract}
\begin{small}
%Two studies with the ATLAS experiment  at the Large Hadron Collider (LHC)  have been performed in this thesis: 
%the search for the neutral MSSM Higgs bosons and an investigation on the prospect for enhancing the sensitivity of this search by using track-based jet
%in alternative to the canonical calorimeter-based jet reconstruction.

In this thesis, a search for the neutral Higgs bosons of the Minimal Supersymmetric extension of the Standard Model has been performed  
with the ATLAS detector  at the Large Hadron Collider (LHC).
The search focuses on  Higgs boson decays
into a pair of $\tau$ leptons which subsequently decays via  $\tau^+ \tau^- \rightarrow e \mu +4\nu$.
The prospects for enhancing the sensitivity of this search by using jet reconstruction based on inner detector tracks
has also been investigated.


The search for the neutral MSSM Higgs bosons $A$, $h$ and $H$ has been performed using proton-proton collision data
at a centre-of-mass energy of 8~TeV corresponding to an integrated luminosity of 20.3$\,\text{fb}^{-1}$. 
%The decay into $\tau$ leptons pair has the highest sensitivity for this search.
%The search is performed for the two most significant production modes, gluon fusion and the production in association with b-quarks. 
To enhance the signal sensitivity,  the   events are split into two mutually exclusive categories
without and with  b-tagged jets indicating the two dominant Higgs boson production modes, via gluon fusion and in association with b-quarks, respectively.
The results are interpreted in terms of the  MSSM $m_h^{mod}$ benchmark scenario. 
No significant excess of events above the estimated Standard Model background has been found. 
Upper  limits have been derived in the plane of the two free MSSM parameters  $m_A$ and $\tan\beta$, where the latter is the ratio of the vacuum expectation values
of the two MSSM Higgs doublets. Values  of $\tan\beta \apprge 10$ are excluded in the mass range $90 < m_A < 200$~GeV. %something on limits
The most significant excess of events with a  local p-value of 2.9\% for the background only hypothesis is observed in the mass rage $250< m_A <300$~GeV,
corresponding to a signal significance of $1.9\,\sigma$.
In addition, less model-dependent upper limits on the cross section for the production of 
 a generic scalar boson $\phi$ with  mass  $m_\phi$ via the  processes $pp \rightarrow b\bar{b}\phi$ and $gg \rightarrow \phi$
have been derived.


The neutral MSSM Higgs boson production in association with b-quarks is characterised by the presence of  low transverse momentum
b-jets. The reconstruction and calibration of low transverse momentum jets based on energy deposits in the calorimeters
 is strongly affected by pile-up effects due to the multiple proton interactions per bunch crossing. 
An alternative approach employing jet reconstruction based on inner detector tracks  
have been  investigated. For jets with low transverse momenta the track-based  reconstruction
provides a higher jet reconstruction efficiency compared to calorimeter-based one and is more suitable  for
the  identification of low momentum  b-jets. This preliminary study shows that the sensitivity of the search for neutral MSSM Higgs bosons,
 produced in association with b-quarks, can be improved by up to a factor of two if track-based jet reconstruction is employed 
instead of the canonical calorimeter-based one. However, additional studies are needed to fully evaluate the
systematic uncertainties of track-based jets reconstruction. Furthermore a dedicated calibration of the b-jet 
identification and mis-identification rates is necessary to complete the study.
 

%This is one of the main reason for sensitivity loss of this search in the category that 
%requires a b-tagged jet. An alternative approach would be to use the so-called track-based jets reconstruction, 
%which relies only on identified inner detector tracks for the jet reconstruction.  The measure of inner detector tracks
%is very precise and is possible to associate them with the corresponding interaction vertex,  
%this feature makes track-based jets more robust against pile-up effects than calorimeter-based jets. 
%A study on the prospect for enhancing the sensitivity of the neutral MSSM 
%An alternative b-jet identification procedure in which the b-tagging algorithm is applied on 
%track-based jets instead on canonical calorimeter jets has been studied in this thesis.
%The calorimeter jets are reconstructed
%from the energy clusters in the calorimeter, while the track-based jets consist of inner detector tracks,
%the precise track information makes the latter considerably more robust to pile-up effects.
%The performance of the b-tagging algorithms on track-based jets is investigated for the first time.
%Systematic uncertainties on track-based jet reconstruction are also studied: one of the major systematic
%uncertainty arises from the mismodelling in simulation of the inner detector material (ID)  budget.
%A novel technique for addressing the ID material budget systematic uncertainty in track-jets energy scale and 
%reconstruction efficiency has been developed. For low transverse momentum track-jets uncertainty due to 
%material budget mismodelling in the energy scale is estimated to range from 2\% to 4\% depending on track-jet momentum and number of 
%associated tracks.
%It was shown that for jets with low transverse momentum the track-jets provide a higher
%b-hadron reconstruction efficiency than calorimeter jets and are more suitable  for
%low $\pt$ b-tagging. The sensitivity to the neutral MSSM Higgs boson produced in association with b-quarks can be improved 
% up to a factor two if track-jet reconstruction is applied instead of calorimeter based one.
%However, to exploit the full power of this technique a dedicated calibration of the 
%b-tagging algorithms is  needed for  track-based jets. 
\end{small}
\end{abstract}
\restoregeometry
