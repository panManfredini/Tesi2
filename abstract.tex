\thispagestyle{empty}
\newgeometry{left=3.7cm, right=3.7cm,bottom=4cm ,top=2cm}
\begin{abstract}
\begin{small}
Two studies with the ATLAS experiment  at the Large Hadron Collider (LHC)  have been performed in this thesis: 
the search for the neutral MSSM Higgs bosons and an investigation on the prospect for enhancing the sensitivity of this search by using track-based jet
in alternative to the canonical calorimeter-based jet reconstruction.

The search for the neutral MSSM Higgs bosons $A$, $h$ and $H$ has been performed using LHC proton-proton collision data at a centre-of-mass energy
of 8~TeV corresponding to an integrated luminosity of 20.3$\,\text{fb}^{-1}$. The search focuses on the Higgs boson decays
into a pair of $\tau$ leptons which subsequently decays via  $\tau^+ \tau^- \rightarrow e \mu +4\nu$.
The decay into $\tau$ leptons pair has the highest sensitivity for this search.
The search is performed for the two most significant production modes, gluon fusion and the production in association with b-quarks. 
To enhance the signal sensitivity,  the  selected events are categorised into two mutually exclusive categories based on the presence
and on the absence of b-tagged jets in the final state.
The results are interpreted in terms of the  MSSM $m_h^{mod}$ benchmark scenario as a function of the free parameter 
 $m_A$ and $\tan\beta$. No significant excess of event above the estimated Standard Model background has been found, 
upper exclusion limits are derived in the $m_A - \tan\beta$ plane for the $m_h^{mod}$ benchmark scenario where values of $\tan\beta \apprge 10$ are excluded 
for the mass range $90 < m_A < 200$, %something on limits
the highest local p-value
is observed in the mass rage $250< m_A <300$~GeV and corresponds to $1.9\,\sigma$.
In addition, less model-dependent exclusion limits are derived on the cross section for 
the production of a generic Higgs boson $\phi$ with  mass  $m_\phi$ via the  processes $pp \rightarrow b\bar{b}\phi$ and $gg \rightarrow \phi$.


The neutral MSSM Higgs bosons produced in association with b-quark are characterised by the presence of relatively low-transverse momentum
jets (originating from b-quarks), the reconstruction and calibration of calorimeter jets for such a low transverse momentum 
suffers strongly of pile-up effects.  An alternative approach employing track-based jet b-tagging is investigated.
For jets with low transverse momentum the track-based jet provides a higher
b-hadron reconstruction efficiency than calorimeter-based  jets and are more suitable  for
low $\pt$ b-tagging. The sensitivity to the neutral MSSM Higgs boson produced in association with b-quarks can be improved 
up to a factor two if track-based jet reconstruction is employed instead of the canonical calorimeter based one.


%This is one of the main reason for sensitivity loss of this search in the category that 
%requires a b-tagged jet. An alternative approach would be to use the so-called track-based jets reconstruction, 
%which relies only on identified inner detector tracks for the jet reconstruction.  The measure of inner detector tracks
%is very precise and is possible to associate them with the corresponding interaction vertex,  
%this feature makes track-based jets more robust against pile-up effects than calorimeter-based jets. 
%A study on the prospect for enhancing the sensitivity of the neutral MSSM 
%An alternative b-jet identification procedure in which the b-tagging algorithm is applied on 
%track-based jets instead on canonical calorimeter jets has been studied in this thesis.
%The calorimeter jets are reconstructed
%from the energy clusters in the calorimeter, while the track-based jets consist of inner detector tracks,
%the precise track information makes the latter considerably more robust to pile-up effects.
%The performance of the b-tagging algorithms on track-based jets is investigated for the first time.
%Systematic uncertainties on track-based jet reconstruction are also studied: one of the major systematic
%uncertainty arises from the mismodelling in simulation of the inner detector material (ID)  budget.
%A novel technique for addressing the ID material budget systematic uncertainty in track-jets energy scale and 
%reconstruction efficiency has been developed. For low transverse momentum track-jets uncertainty due to 
%material budget mismodelling in the energy scale is estimated to range from 2\% to 4\% depending on track-jet momentum and number of 
%associated tracks.
%It was shown that for jets with low transverse momentum the track-jets provide a higher
%b-hadron reconstruction efficiency than calorimeter jets and are more suitable  for
%low $\pt$ b-tagging. The sensitivity to the neutral MSSM Higgs boson produced in association with b-quarks can be improved 
% up to a factor two if track-jet reconstruction is applied instead of calorimeter based one.
%However, to exploit the full power of this technique a dedicated calibration of the 
%b-tagging algorithms is  needed for  track-based jets. 
\end{small}
\end{abstract}
\restoregeometry
