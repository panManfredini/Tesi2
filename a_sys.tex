\section{Systematic Uncertainties}
\label{sec:Systematics}

In this section the systematic uncertainties for the measurement of this thesis are discussed.
To account for differences in the actual and simulated detector response, corrections are applied at the level of
object reconstruction  and of event selection as described in Chapter~\ref{chap:obj}. 
The detector-related systematic uncertainties due to these corrections are addressed in Section~\ref{sec:sys:sys_det}. 
For all processes whose contributions are predicted by simulation, also theoretical
uncertainties related to cross-section calculation and to the acceptance of the selection criteria 
have been taken into account as  described in section~\ref{sec:sys_theory}$\,.$
The Systematic uncertainties due to  background measurements using dedicated control data samples
are described in Sections~\ref{sec:embsys} and~\ref{sec:qcdsys}$\,.$

Each source of systematic uncertainty can contribute separately to the uncertainties in the
final event yield and in the shape of the $\mmc$
distribution which is used as final discriminating variable in the statistical interpretation of the data. Systematic uncertainties
affecting the shape of the mass distribution are
documented in Appendix~\ref{appendix:shapeNPs}. Uncertainties in the shape of the \mmc  distribution 
are found to be significant only for the embedded sample in the b-vetoed category,
for all other  backgrounds these uncertainties are found to be negligible.
Systematic uncertainties which do not affect the
shape of the mass distribution and have an impact on the event yield of less than 0.5\% for each process are 
neglected.
%hey in fact do not have any significant effect on the final expected limits.


\subsection{Detector-Related Systematic Uncertainties}
\label{sec:sys:sys_det}
Systematic uncertainties related to object reconstruction and event-by-event 
corrections are derived from the calibration measurements of the relevant parameters. These 
parameters correspond to a nuisance parameter in the probability model used for the  statistical interpretation
of the data as described in Section~\ref{sec:result}.
Each parameter is varied independently by one standard deviation  according to its measured
uncertainty, the impact on the event  yield is evaluated for each simulated signal and background sample.
In the following the detector-related uncertainties are  described in more detail.
Tables~\ref{tab:ExpSys:bveto} and~\ref{tab:ExpSys:btag}  summarise the systematic uncertainties in the 
predicted event yields. 


\paragraph{Luminosity}
The integrated luminosity recorded during 2012 by the ATLAS detector at centre-of-mass energy of 8~TeV is measured 
to be $20.3 ~ fb^{-1}$ \cite{luminosity} with an uncertainty  of  2.8\%.

\paragraph{Pileup}
Simulated events are re-weighted to reproduce the average number of interactions per bunch crossing $<\mu>$ as seen in data. 
Those event weights have an uncertainty which is propagated to each simulated sample.
%It has been seen
%that a proper description of the minimum bias vertex multiplicity is obtained if the $<\mu>$ 
%value in MC is first scaled by a factor of $1.11 \pm 0.03$ before re-weighting to match data. The uncertainty
%on this value is taken as a systematic uncertainty for the analysis.

\paragraph{The Trigger Efficiency}
is corrected in simulation to match on  average the one observed in data. The correction weights and their uncertainties 
are evaluated as a function of $\pt$ and $\eta$ of the leptons. 
Systematic uncertainties in the single-electron and the electron-plus-muon trigger efficiencies are taken into account 
independently and are in the range of 1-2\%.

For the embedded $\Ztt$ sample, the trigger is emulated by applying weights  depending on the  $\pt$ and $\eta$ of the leptons. 
These weights  have similar uncertainties as the ones described above. Trigger efficiency uncertainties for the embedded sample 
 are considered uncorrelated with the ones of other samples.

\paragraph{Electrons}
Two sources of uncertainty in the electron reconstruction are considered:
the first  related to electron identification and reconstruction efficiencies ("Electron ID") which range approximately from 1-2\%
depending on the transverse momentum of the electron, 
the second related to electron energy scale  and resolution, both in the range of 0.3-3\% depending on the pseudorapidity of the electron. 
The energy scale uncertainties are described by six nuisance parameters~\cite{eleEnergy}.
Only a few of them gives a non-negligible contribution to the systematic error. Two of them 
affect the shape of the \mmc distribution and are considered independently: one is related to the electron  
momentum measurement from  $Z \rightarrow ee$ data ("Electron Zee") 
and the other related to the reconstruction of low momentum electrons ("Electron LOWPT"). 
All other uncertainties related to energy scale and resolution are combined quadratically ("Electron E").

\paragraph{Muons}
The uncertainty in the  muon identification efficiency depends on the charge and momentum of the muon.
Typically these uncertainties are on the order of a fraction of percent, and are referred to as "Muon ID". 
The uncertainties in the muon energy scale and resolution are considered independently for the inner detector 
and muon spectrometer measurements and are then  added in quadrature ("Muon E").

\paragraph{$\tau$-Jets}
Jets from  hadronically decaying $\tau$ leptons are vetoed in this analysis. Uncertainties in both the $\tau$-jet energy scale 
and the identification efficiency have been investigated and were found to be negligible.

\paragraph{Jets}
The systematic uncertainties in the jet energy scale (JES) are described by multiple sets of nuisance parameters~\cite{JES}
related to different effects and jet energy components, for example  pileup effects and the flavour composition of the jets. 
The overall uncertainty in the JES ranges from  3 to 7\% depending on $\pt$ and $\eta$ of the jet. 
The overall impact of the JES uncertainty on the event yields as shown in Tables~\ref{tab:ExpSys:bveto} and~\ref{tab:ExpSys:btag}
is determined adding all contribution in quadrature, while in the statistical interpretation of data
those uncertainties are considered independently.
Systematic uncertainties due to the jet energy resolution ("Jet Resolution") are obtained by smearing the jet energy 
according to the measured  uncertainty which ranges from 10 to 20\% depending on the jet direction.
%The recommendation \cite{TWIKI_JETMET} to propagate each individual 
% source of uncertainty through the full analysis is followed. In this analysis the reduced set of fourteen uncertainties, know as 
% \verb=InsituJES2012_14NP=, have been considered: these include two uncertainties for eta inter calibration, four pile-up 
% uncertainties, a high-$\pt$ uncertainty and an uncertainty for MC non-closure. Of these, only the terms "JES Effective 1", 
% "JES Effective 2" and "JES Effective 3", the pileup uncertainty as a function of the number of primary vertices ("JES Pileup-NPV") and the uncertainty due to the jet area ("JES Pileup-Rho") have a significant effect on the event yields. Additionally, 
% the uncertainty related to the fraction of quark to gluon jets ("JES Flav. Comp") and on the different response to them ("JES Flav. Resp.") are considered. The additional uncertainty assigned to the b-jet energy scale, referred as "JES B", is also 
% significant to this analysis. Uncertainties related to theory and modelling ("JES EtaModelling") also contribute to this 
% analysis. Finally the effect of the jet energy resolution ("JES Resolution") is evaluated applying a smearing to the jets, the 
% resulting effect on the yield is symmetrised.

\paragraph{b-Tagging}  Corrections are applied to simulation
to match the b-tagging efficiency observed  in data. Uncertainties on the knowledge 
of the b-tagging efficiencies for the 70\% $\epsilon_b^{\ttbar}$ working point of the MV1 b-tagger are
considered~\cite{BtaggingScaleFactors,BtaggingScaleFactorsNew}. These uncertainties range from 5-10\%  depending on the $\pt$ of the jet. 
The effect of those uncertainties is evaluated independently for the
 b-quark, c-quark and light or gluon initiated jets and referred to respectively 
 as "B  Eff", "C Eff" and "L Eff". The tagging and mis-tagging efficiency uncertainties 
 are considered to be fully anti-correlated. 

\paragraph{Missing Transverse Energy}
The effect of the energy scale uncertainties for all  physics objects is propagated to the \met calculation.
In addition, uncertainty on the energy scale and resolution due to the remaining unassociated 
calorimeter energy deposits, the ``soft-terms'', is considered and estimated to be of the order of 10\%~\cite{ETMISS}. 
\met uncertainties are independently propagated through the analysis and are
added in quadrature, this final term is referred to as the "MET" uncertainty.


%\paragraph{Summary} A summary of the effect of the experimental and theoretical systematic uncertainties on signal and background yields for the b-tag and b-veto channels are shown in Table~\ref{tab:ExpSys:btag} and Table~\ref{tab:ExpSys:bveto}, respectively. It should 
%be noted that the gluon fusion  signal sample suffers of poor statistics in the b-tag category. 
%Hence, some of the yield differences reported are statistically dominated for these samples.
%%	As a solution the mean value between all the sample mass point can be taken as measure of the single systematic, 
%However the gluon fusion production mode has a negligible contribution in b-tag category.
\begin{table}[!htp]
  \centering
  \begin{tabular}{lccccc}
    \hline\hline
      	      		   \multicolumn{6}{c}{  Systematic uncertainties in event yields (\%), b-vetoed  category}  \\
     \hline
      Source             & Signal bbH & Signal ggH & \Ztautau &  Top 	& Other	 \\
    \hline
Electron ID  		 &2.4		   &2.3		     &2.9 (\bf{s})	     &1.4	&1.6	 \\
Electron E.	  	 &0.4		   &0.5		     &0.4	     &0.5	&0.9	 \\
Electron LOWPT	  	 &0.3		   &0.5		     &0.4 (\bf{s})	     &0.0	&1.2  \\ 
Electron Zee	  	 &0.4		   &0.4		     &0.4 (\bf{s})	     &0.1	&0.3	 \\
Muon ID 		 &0.3		   &0.3		     &0.3	     &0.3	&0.3	 \\
Muon E.		  	 &0.1		   &0.1		     &0.1	     &0.5	&0.5	 \\
Trigger Single	Ele.  	 &0.6		   &0.6		     &0.5	     &0.9	&0.9	 \\
Trigger Dilep.	  	 &1.0		   &1.0		     &1.3	     &0.2	&0.3	 \\
Embedding MFS	  	 &-		   &-		     &0.1 (\bf{s})   &-		&-	 \\
Embedding Iso.	  	 &-		   &-		     &0.0 (\bf{s})   &-		&-	 \\
JES		  	 &0.6		   &0.7		     &-		     &1.0	&1.2	 \\
Jet Resolution  	 &0.5		   &0.3		     &-		     &0.6	&0.3	 \\
B Eff		  	 &1.8		   &0.0		     &-		     &12.0	&0.8	 \\
C Eff	  		 &0.0		   &0.1		     &-		     &0.1	&0.0	 \\
L Eff	  		 &0.0		   &0.1		     &-		     &0.2 	&0.1	 \\
Pileup			 &0.5		   &0.8		     &0.4	     &0.3	&0.3	 \\
MET  		  	 &0.2		   &0.8 	     &0.1	     &0.2	&0.5	 \\
%Acceptance		 &		   &		     &		     &		&	  \\
%Cross Section	  	 &-		   &-		     &5.0	     &5.5	&5.9	 \\
Luminosity	  	 &2.8 		   &2.8	 	     &2.8 	     &2.8 	&2.8 	 \\

    \hline
    \hline
  \end{tabular}
  \caption{Experimental systematic uncertainties in the event yields of the different
	simulated samples in the b-vetoed  category. "Other" refers to the sum of all remaining background contribution from: $\Wlnu$, 
	dibosons, $\Zll$ and single top quark processes. The Signal produced in association with b-quarks  and via 
	gluon fusion is considered separately assuming $m_{A}=150$ GeV and $\tan\beta=20$ in the $m_h^{mod}$ scenario. 
	Uncertainty influencing the shape of the  \mmc distribution labelled with (\textbf{s}).} 
 \label{tab:ExpSys:bveto}
\end{table}

	
\begin{table}[tp]
  \centering
  \begin{tabular}{lccccc}
    \hline\hline
      	      		   \multicolumn{6}{c}{  Systematic uncertainties in event yields (\%), b-tagged  category}  \\
     \hline
      Source             & Signal bbH 	   & Signal ggH      & \Ztautau      &  Top 	& Other	 \\
    \hline
Electron ID  		 &2.3		   &2.6		     &	2.8          &1.8	&2.0	 \\
Electron E	  	 &0.7		   &1.2		     &0.5	     &0.5	&0.9	 \\
Electron LOWPT	  	 &0.4		   &0.0		     &0.4	     &0.1	&0.4	 \\ 
Electron Zee	  	 &0.3		   &0.6		     &0.4	     &0.6	&0.5	 \\
Muon ID 		 &0.3		   &0.3	   	     &0.3	     &0.3	&0.3	 \\
Muon E		  	 &0.5		   &0.8		     &0.1	     &0.1	&0.2	 \\
Trigger Single	Ele.  	 &0.7		   &0.5		     &0.5	     &0.8	&0.8	 \\
Trigger Dilepton  	 &1.0		   &1.2		     &1.4	     &0.6	&0.6	 \\
Embedding MFS	  	 &-		   &-		     &0.0	     &-		&-	 \\
Embedding Iso.	  	 &-		   &-		     &1.3	     &-		&-	 \\
JES		  	 &2.7		   &7.3		     &-		     &10.0	&7.0	 \\
Jet Resolution	  	 &1.4		   &6.3		     &-		     &2.9	&3.0	 \\
B Eff		  	 &10.2		   &3.1		     &-		     &2.6	&5.0	 \\
C Eff		  	 &0.2		   &4.3		     &-		     &0.0	&1.2	 \\
L Eff		  	 &0.4		   &8.0		     &-		     &0.1	&1.2	 \\
Pileup			 &0.4		   &0.7		     &0.4	     &0.4	&0.9	 \\
MET 		  	 &0.7		   &0.5 	     &0.2	     &1.0	&1.2	 \\
%Acceptance		 &		   &		     &		     &		&	  \\
%Cross Section	  	 &-		   &-		     &5.0	     &5.5	&7.1	 \\
Luminosity	  	 &2.8 		   &2.8	 	     &2.8 	     &2.8 	&2.8 	 \\

    \hline
    \hline
  \end{tabular}
  \caption{Experimental systematic uncertainties in the event yields of the different
	simulated samples in the b-tagged category. "Other" refers to the sum of all remaining background contribution from: $\Wlnu$, 
	dibosons, $\Zll$ and single top quark processes. The Signal produced in association with b-quarks  and via 
	gluon fusion is considered separately assuming $m_{A}=150$ GeV and $\tan\beta=20$ in the $m_h^{mod}$ scenario.} 

  \label{tab:ExpSys:btag}
\end{table}




\subsection{Theoretical Uncertainties}
\label{sec:sys_theory}
%\subsection{Simulated Cross-Section Uncertainties}


\begin{table} [!t]
\caption{Cross-section uncertainties for signal and background  processes assuming $\tan\beta = 20$ for all signal samples.} \vspace{3mm}
\centering
\begin{tabular}{c c c }
\hline
\hline
Process 					&  Generator		& Uncertainty (\%) \\ [0.5ex]
\hline
 $Z \rightarrow \tau\tau / ee /\mu\mu$ 		& ALPGEN		&  $\pm 5$ \\ [0.5ex]
 \ttbar						& POWHEG		&  $\pm 5.5$\\ [0.5ex]
 $W  \rightarrow \tau\nu / e\nu /\mu\nu$	& ALPGEN		&  $\pm  5$ \\ [0.5ex]
 single top 					& MC@NLO / AcerMC	&  $\pm 7 ~ / ~\pm 13 $ \\ [0.5ex]
 dibosons 					& HERWIG		&  $\pm 6 $ 		\\ [0.8ex]
 $bbA$/$h$/$H$  ($m_{A} \ge 120$~GeV)     	& SHERPA		&  $_{-20}^{+9}$	\\ [0.8ex]
 $bbA$/$h$/$H$  ($m_{A} =   110$~GeV)     	& SHERPA		&  $_{-25}^{+9}$	\\ [0.8ex]
 $bbA$/$h$/$H$  ($m_{A} =   100$~GeV)     	& SHERPA		&  $_{-28}^{+9}$	\\ [0.8ex]
 $bbA$/$h$/$H$  ($m_{A} =    90$~GeV)     	& SHERPA		&  $_{-30}^{+9}$	\\ [0.8ex]
 $ggA$/$h$/$H$  ($m_{A} \le 300$~GeV)     	& POWHEG		&  $\pm 15$		\\[0.5ex]
\hline \hline 
\end{tabular}
\label{table:sys_xsec}
\end{table}


Uncertainties on the cross-sections that have been used to normalise
the contribution of simulated samples to the integrated luminosity of analyzed data are reported in
Table~\ref{table:sys_xsec}. These
uncertainties include contributions due to parton distribution
functions (PDFs), the choice of the value of the strong coupling constant,
the renormalisation and factorisation scales.  Furthermore, the
uncertainties on the signal cross-section depend on the  $\tan\beta$ value, the type of  
Higgs boson ($A$/$h$/$H$) and its mass.

The  systematic uncertainties due to Monte Carlo tuning
parameters for the description of the  underlying event,
of lepton kinematical properties and parton density functions have been studied.
Since the distribution of the invariant mass of all visible $\tau$ lepton decay 
products is found not to be affected by these systematic uncertainties,
% as an example see
%Figure~\ref{fig:theory_mass}, 
only the impact on the acceptance is considered.
The acceptance uncertainties for the simulated ALPGEN $\Ztautau$  sample, which is 
used for the normalisation of the embedded sample, 
are estimated at the common selection stage to be 4\% \cite{ATLASLimit}.
%\footnote{A bit old result, to be reviewed. Depends also on the
%embedding choice}
Since additional selection criteria  are applied directly to the embedded sample, 
no further acceptance uncertainties are considered. Acceptance uncertainties in the yield of 
simulated $t\bar{t}$ events are estimated to be  2\%~\cite{ttbaremu}. %\textcolor{red}{still to be added}.
%evaluated
%\footnote{also here is not totally clear yet} 
%by the difference between MC@NLO and POWHEG in
%the data-driven background estimate. 
%For other, single and dibosons
%production as well as single top production a 2\% uncertainty is
%assumed.
The acceptance uncertainties in diboson and single top quark production are estimated to be 2\%~\cite{ATLASLimit}.
%\footnote{preliminary, this is following LEP-Had note, we
%  should cite some previous result here.}.

Uncertainties in the signal acceptance have been estimated using signal 
samples simulated with different generator parameters. The impact on the selection
 of leptons, $\tau$-jets  and jets is evaluated at the particle level, prior to 
simulation of the detector response. This truth-level study is implemented within the Rivet framework
\cite{RIVET}, where the b-tagging is performed by identifying the b-quarks and applying
 weights according to the measured ATLAS b-tagging
efficiencies \cite{BtaggingScaleFactors}. The variation of the acceptance
with respect to the nominal Monte Carlo tune has  been considered as
a source of systematic uncertainty. The total signal acceptance 
uncertainty varies from 4\% to 30\% depending on $m_A$, on the production process 
and event category.

%The acceptance uncertainties of the two signal production modes are
%evaluated separately because of the use of different generators for each. For
%b-quark associated production, generated with SHERPA,
%the CKKW matching parameter $Q_{cut}$ has been varied from its default
%of $\sqrt{20 ~ GeV/E_{CMS}}$ to values of $\sqrt{15 ~ GeV/E_{CMS}}$
%and $\sqrt{30 ~ GeV/E_{CMS}}$. The factorisation scale was varied up
%and down by a factor of two and the renormalisation scale by a factor of
%10\%. Uncertainties due to the PDFs were determined by taking the RMS
%of the acceptance of the 52 error sets of the CT10 PDF set.  These
%effects are summarised in Table~\ref{table:sys_bba}. For a total
%uncertainty, all effects are summed in quadrature giving a total
%uncertainties that varies from 4\% to 30\% depending on $\mA$ and on the
%analysis category.  For gluon fusion production, generated with POWHEG
%and Pythia 8, the initial and final state
%radiation uncertainties were varied up and down, and the
%renormalisation and factorisation scales were varied simultaneously
%(the renormalisation scale by 10\% and the factorisation scale by factor 2\%).
%PDFs uncertainties were handled in the same way as for the $b$-quark
%associated production.  These variations are summarised in Table
%\ref{table:sys_gga}.  The uncertainties shown in Tables
%\ref{table:sys_bba} and \ref{table:sys_gga} are based on samples with
%$m_{A} = 120 \text{ GeV}$.  Results for the mass points 90, 200
%and 300 GeV are shown in Appendix \ref{appendix:additional} of this note.
 


%\begin{table}[tdp]
%  \begin{center}
%   \label{table:sys_bba}
%   \begin{tabular}{lccccc}
%\hline \hline
% Event yields       & b-tag deviation [\%]  & b-veto deviation [\%] \\
%\hline
%CKKW down &    $ -3.1 \pm 0.9 $ &      $ 0.4 \pm 0.4 $ \\
%CKKW up &     $ -8.3 \pm 0.9 $ &      $ 2.9 \pm 0.4 $ \\
%Fac. scale down &    $ 15.5 \pm 1.0 $ &   $ -4.2 \pm 0.4 $ \\
%Fac. scale up &     $ -19.8 \pm 0.8 $ &    $ 5.6 \pm 0.4 $ \\
%Ren. scale down &      $ 0.4 \pm 0.9 $ &     $ -0.3 \pm 0.4 $ \\
%Ren. scale up &     $ 0.8 \pm 0.9 $ &      $ 0.5 \pm 0.4 $ \\
%PDF &     $\pm 0.1 $ 		& $\pm 0.2 $ \\
%\hline
%Total  (up) &     $ 13.5 \pm 1.6$ &                    $ 6.3 \pm 0.8$ \\
%Total  (down) &     $ -21.7 \pm 1.5$ &                    $ -4.2 \pm 0.6$ \\ 
%\hline \hline
%	\end{tabular}
%   \caption{Signal acceptances for several systematic deviations of the theory parameters contributing to the b-quark associated production of Higgs bosons. The different variations are added in quadrature to a total uncertainty on the signal acceptance in the b-tag and b-veto channels.}
%  \end{center}
%\end{table}
%
%
%\begin{table}[tdp]
%  \begin{center}
%   \label{table:sys_gga}
%    \begin{tabular}{lccccc}
 %   \hline \hline
% Event yields      &   b-tag deviation [\%] &   b-veto deviation [\%] \\
%\hline
%ISR up & $ 20.3 \pm 8.1 $ 		& $ -1.2 \pm 0.6 $ \\
%ISR down & $ 3.6 \pm 7.2 $ 		& $ 0.4 \pm 0.6 $ \\
%FSR up & $ 16.6 \pm 7.8 $ 		& $ -0.2 \pm 0.6 $ \\
%FSR down & $ -3.6 \pm 6.8 $	 	& $ -0.7 \pm 0.6 $ \\
%Ren./Fac. scales up & $ 9.4 \pm 7.5 $ 	& $ 0.0 \pm 0.6 $ \\
%Ren./Fac. scales down & $ 2.5 \pm 7.1 $ & $ -0.5 \pm 0.6$ \\
%PDF &     $\pm 0.0 $ &$\pm  0.1 $ \\
%\hline
%Total (up) &    $ 28.2 \pm 16.9 $ &     $ 0.4 \pm 0.6$ \\
%Total (down) &    $ -3.6 \pm 6.8 $ &     $ -1.5 \pm 1.2$ \\ 
%\hline \hline
%	\end{tabular}
%   \caption{Signal acceptances for several systematic deviations of the theory parameters contributing to the Higgs boson production through gluon fusion. The different variations are added in quadrature to a total uncertainty on the signal acceptance in the b-tag and b-veto channels.}
%  \end{center}
%\end{table}


%%%%%%%%%%%%%%%%%%%%%%%%%%Figure Lorentz%%%%%%%%%%%%%%%%%%%%%%
%\begin{figure}[tp]
%\begin{center}
%\includegraphics[width=0.65\textwidth]{figure/facs_mll_bveto2}
%\end{center}
%\caption{ Distribution of the invariant mass of all visible $\tau$-lepton decay products 
%for different choices of the factorisation scale. The shown distribution is 
%for gluon fusion produced signal in the b-vetoed event category.}
%\label{fig:theory_mass}
%\end{figure}
%


\subsection{Systematic Uncertainties of  \Ztautau Embedded Sample}\label{sec:embsys}
\begin{figure}[!t]
	\begin{center}
	\includegraphics[width=0.45\textwidth]{figure/systematics/emb_sys_veto_MFS.pdf}
	\includegraphics[width=0.45\textwidth]{figure/systematics/emb_sys_veto_iso.pdf}
	\end{center}
	\caption{Impact of EMB\_MFS (left) and EMB\_ISO (right) systematic uncertainties in the $\mmc$ distribution of  embedded events.
	Significant uncertainty have been found only in the b-vetoed category.}
	\label{fig:EMBMFS}
\end{figure}

An important element of the embedding method is the subtraction of the 
calorimeter cells associated with the muons in the original \Zmumu event and their substitution with those from the simulated $\tau$ lepton
decays. To make a conservative estimate of the systematic uncertainty on this procedure, 
the energy of the subtracted cells is scaled up or down by 30\%. The analysis is repeated with those modified 
samples and the relative uncertainty is referred as "EMB\_MFS". This uncertainty affects mainly the shape of the \mmc mass 
distribution as shown in Figure~\ref{fig:EMBMFS}.



%\begin{figure}[htp]
%     \begin{center}
%
%        \subfigure[]{%
%            \label{fig:mvis}
%            \includegraphics[width=0.45\textwidth]{figure/distributions/NP_Shape_EmbMFS_BVeto_mmc.pdf}
%	}
%	
%        \subfigure[]{%
%            \label{fig:mmc}
%            \includegraphics[width=0.45\textwidth]{figure/distributions/NP_Shape_EmbIso_BVeto_mmc.pdf}
%	}
%
%    \end{center}
%    \caption{Effect on the \mmc distribution of the embedding sample due to (a) the EMB\_MFS and (b) embedding isolation systematics. The plots are made after the full b-veto category selection.}
%   \label{fig:embeddingShapeNPs}
%\end{figure}

In the sample of  \Zmumu candidates used for the embedding,  only a loose requirement on the  muon track isolation is used.
A different  muon isolation requirement may affect the selected sample by modifying the topology of the event, 
% since the requirement is indirectly acting also on the muon \PT, 
changing the contamination with other processes or the activity in the calorimeter. 
To estimate  the importance of these effects in the
embedded sample, the muon isolation criteria used for  the original \Zmumu sample are tightened,
a looser requirement have a  small impact due to the isolation requirements at the trigger level.
The resulting uncertainty, referred to as "EMB\_ISO", affects both the event yield and the shape of 
the \mmc  distribution of the embedded sample as shown in Figure~\ref{fig:EMBMFS}. 

Finally, because the normalisation of the embedded sample is determined from the ALPGEN simulation, 
the uncertainties related to the cross section calculation and the luminosity are assigned. In addition,
all the discussed detector-related systematic uncertainties affecting the decay products of the simulated $\tau$ lepton 
decays are propagated to the embedded sample.

 
%\begin{figure}[tp]
%	\begin{center}
%	\includegraphics[width=0.49\textwidth]{figure/systematics/emb_sys_BtagFull_Iso.png}
%	\end{center}
%	\caption{embedding Isolation systematic uncertainty impact $\mmc$.}
%	\label{fig:EMBISO}
%\end{figure}

\subsection{QCD Multi-Jet Systematic Uncertainties}\label{sec:qcdsys}

\begin{table} [!tp]
	  \caption{Comparison between the transition factor \rqcd, $\rqcd^{AB}$ and $R_{QCD}^{iso}$ after the common selection 
	and after b-tag or b-veto requirements on jets. 
	The value of $\rqcd^{iso}$ is calculated for a  lepton isolation threshold which is twice the nominal value,
	while  for  $\rqcd^{AB}$ and \rqcd the nominal values are given. The uncertainties are statistical only.}
	\vspace{3mm}

	\begin{center}
	\begin{tabular}{l  c c c }
%%%%%%%%%%%%%%%%%%%%%%%%%%%%%%%%%%%%%%%%%%%%%%%%%%%%%%
\hline 
\hline
Selection  		&  \rqcd  			&  $\rqcd^{AB}$  		&  $R_{QCD}^{iso}$ \\ 
\hline
Common selection 		&   1.929 $\pm$     0.004	&	2.12 $\pm$ 0.17		&	2.22 $\pm$ 0.16	\\
No b-tagged jets		&  1.965   $\pm$   0.005    	& 2.10   $\pm$	0.16 		&	2.22 $\pm$ 0.16	\\
Exactly one b-tagged jet	&  1.78    $\pm$   0.02 	& 1.9   $\pm$	0.9 		&	2.0  $\pm$ 0.8	\\
\hline
\hline
%%%%%%%%%%%%%%%%%%%%%%%%%%%%%%%%%%%%%%%%%%%%%%%%%%%%%%
	\end{tabular}
	\label{table:MCsub}
	\end{center}
\end{table}

The QCD multi-jet background is estimated via the ABCD method as
described in Section~\ref{sec:qcd}. This technique relies strongly on
the assumption that the lepton isolation variables are uncorrelated with the product of the
charge signs of the two leptons. Systematic uncertainties
are evaluated to take into account possible deviations from this assumption.
The dependence of the ratio  \rqcd on the lepton isolation criteria is evaluated and then
compared to the uncertainties obtained with auxiliary measurements. 




Figure~\ref{fig:os_ss_ratio} shows the \rqcd factor, i.e. the ratio of the QCD background 
yields in data samples C and D, as a function of a sliding lepton isolation threshold relative to the 
nominal analysis selection.
%correlation is clearly visible. 
The expected contamination with  non-QCD background processes is subtracted from  samples C and D.
To estimate the uncertainty in the value of \rqcd,  another transfer factor is defined as $R_{QCD}^{iso}  = N_{\hat{A}} / N_{\hat{B}}$,
where  $\hat{A}$ and $\hat{B}$  are ``semi-isolated'' OS and SS samples with  the requirement on  the lepton isolation to be  larger 
than the nominal value, 
but smaller than the sliding threshold defined by the $X$-axis of the plot. Also here the non-QCD contributions are subtracted.
%$R_{QCD}^{iso}$ is an attempt to calculate a best estimate for the QCD transfer factor between isolated samples,
%which is in definitive the goal of the ABCD method. 
The semi-isolated samples $\hat{A}$ and $\hat{B}$ have been chosen  
because of the large contamination of the samples A and B with non-QCD background and  possible signal processes. 
Figure~\ref{fig:os_ss_ratio} shows $R_{QCD}^{iso}$ as a function of the relative lepton isolation threshold.
The difference between \rqcd and $R_{QCD}^{iso} $ in the vicinity of the nominal isolation threshold
is then assigned as a systematic uncertainty in \rqcd. For a lepton isolation threshold of 
twice the nominal required value, a systematic uncertainty of 15\% is found.
The result are shown  in Figure~\ref{fig:os_ss_ratio} after the common selection, similar results are obtained after the full 
selection of the two event categories as shown in Appendix~\ref{appendix:qcd_additional}.

\begin{figure}[p]
	\begin{center}
	\includegraphics[width=0.50\textwidth]{figure/systematics/QCD_presel_SYS.pdf}
	\end{center}
	\caption{Transfer factors \rqcd and $R_{QCD}^{iso}$  (see text) as a function of the sliding lepton isolation 
	thresholds. The thresholds are varied in percentages relative to the nominal lepton isolation threshold (value of zero on the plot).
	The common selection are applied.
	%As an example the point at 100\% in the plot corresponds
	%to $\rqcd$ evaluated by increasing the isolation requirement by 100\% respect to the standard cut value.
%	The red points show the anti-isolated scale factor $\rqcd$, i.e. the ratio between samples C and D.
%	 The black points show the isolated scale factor, which is defined as the ratio between sample $\hat{A}$ and $\hat{B}$, 
%	 where the leptons have isolation values larger than the nominal value but smaller
%	 than the sliding cut on X axis.
%	 taking the same example point at 100\%, than the double of the standard cut value.
	 }
	\label{fig:os_ss_ratio}
\end{figure}

\begin{figure}[p]
	\begin{center}
	\includegraphics[width=0.45\textwidth]{figure/systematics/qcd_shape_tag.pdf}
	\includegraphics[width=0.45\textwidth]{figure/systematics/qcd_shape_veto.pdf}
	\end{center}
	\caption{Differences in the shape of the invariant \mmc mass distribution in data samples  C and D shown 
	separately for the b-tagged  and b-vetoed event categories. The data samples C and D are normalised to the same 
	number of events.}
	\label{fig:qcd_shape_unc}
\end{figure}

%for the definition of lepton isolations used in this analysis, the ratio is calculated in samples where the isolation requirements are reversed. Due to a high contamination of signal and non-QCD backgrounds, "semi-isolated" OS and SS samples are additionally defined, where the lepton isolation is larger than the standard requirement, but less than a sliding cut. These samples are labelled $\hat{A}$ and $\hat{B}$  for the semi-isolated OS and SS samples, respectively, and hence we can define $R_{QCD}^{iso}  = \hat{A} / \hat{B}$. The difference between \rqcd and $R_{QCD}^{iso} $ in the vicinity of our standard cut value is then assigned as a systematic uncertainty on \rqcd. Using the point where the cuts on the lepton isolation are twice their standard values, i.e.. the $x=100\%$ point on the graph, a systematic uncertainty of 15\% is found.

%An additional method considers calculating $\rqcd^{AB}$ as the ratio between the estimated QCD contributions in sample A and B.
%These samples, however, suffers of large contribution of non-QCD background and possible signal contamination, 
%this method is then only used as a cross check. Table~\ref{table:MCsub} shows a comparison between \rqcd and  $\rqcd^{AB}$
%for the two category at an early stage of the cut-flow where signal contamination is negligible, agreement is seen within statistical uncertainties.

For validation of the results described above an additional measurement is performed.
The transfer factor $\rqcd^{AB}$ is calculated as the ratio of the estimated QCD multi-jet contributions of the samples A 
and B (instead of C and D). The non-QCD contributions are subtracted. Due to the large contribution of non-QCD background 
along with small numbers of observed events and signal contamination, this measurement is only used as cross check. Table~\ref{table:MCsub} shows 
a comparison of \rqcd, $\rqcd^{iso}$ and $\rqcd^{AB}$ after the common selection and after requiring
or vetoing the presence of b-tagged jets.  At these selection stages the signal contamination is negligible. 
Good agreement is found between all three results. 

%\footnote{This effect is maybe due to the use of a \PT dependent isolation variable that effects the quark-gluon fraction.}.
%Expectation for non-QCD backgrounds are subtracted as usual. 
%This effect however doesn't tell anything on the uncertainty of our
%measure of \rqcd, we want to measure instead what is the discrepancy (for each
%chosen isolation cut value) between \rqcd and the same factor calculated 
%flipping isolation requirements, i.e. using the isolated samples A and B, we call this factor $R_{QCD}^{iso}$. 
%Due to the high contamination of non-QCD backgrounds and signal in these samples we then define:
%OS and SS isolated samples $\hat{A}$ and $\hat{B}$ in which the leptons isolation
%should be greater than the standard value but less of predefined quantity on the x axis 
%of the graph (black curve). In definitive we have $R_{QCD}^{iso}  = \hat{A} / \hat{B}$.
%We then assign as a systematics uncertainty the difference between the two curves (red and black) in the vicinity of our
%standard cut value (we use the point where the cut value is doubled, 100\% in the graph because 
%of statistical fluctuation), our estimate of the systematics uncertainty on \rqcd is then 15\%.
%The plot in Figure~\ref{fig:os_ss_ratio} is made at common selection level, similar plots using the full selection
%for the two categories are in Appendix~\ref{appendix:qcd_additional}.

%An additional method used as a crosscheck relies on the definition of "real-$\rqcd$" as the pure ratio between sample A and B (non-QCD background
%estimate is subtracted from data in each samples), this would be the exact factor 
%that allows you to extrapolate yield from sample B to SR, however it suffer of contamination by non-QCD backgrounds
%and lack of statistics.
%Table~\ref{table:MCsub} shows comparison between \rqcd and real-\rqcd
%for the two category at an early stage of the cut-flow where signal contamination is negligible.
%Discrepancies are within statistical uncertainty and underline that an assignment of a  15\%
%uncertainty to the \rqcd factor is conservative.



The shape of the \mmc distribution differs between the non-isolated  OS and SS lepton samples C and D as shown in Figure~\ref{fig:qcd_shape_unc},
the ratio \rqcd is then dependent on the \mmc distribution.
For the mass range in which the QCD multi-jet background is relevant ( $\mmc  < 150$~GeV),
the size of this effect is within the uncertainty in  \rqcd,  hence no correction is applied to the shape of the mass distribution
in sample B. However, it is assumed that there could be the same 
shape difference in the isolated lepton samples. Thus, a shape uncertainty is  in the mass distribution in 
sample B is taken into account. Further 
shape uncertainties due to non-QCD background subtraction are found to be negligible. The uncertainty due to the use of the isolation 
requirement at the trigger level is also found to be negligible as discussed in Appendix~\ref{appendix:qcd}\,.






