\newgeometry{left=4.5cm, right=4.5cm,bottom=4cm, top=4cm}

\chapter{The Higgs Bosons and the MSSM}

 \vspace{2cm}

This chapter is devoted to introduce the theoretical background to the experimental search presented in this thesis.
A brief overview of the Standard Model of particle physisics is given in Section~\ref{sec:SM} based on Reference~\cite{Altarelli}. 
Among all the extesion of the Standard Model, the Minimal Supersymmetric 
extension of the Standard Model (MSSM) is a theoretically favoured scenario as one of the most predictive
framework beyond the Standard Model, it is introduced in Section~\ref{sec:MSSM} with focus on its Higgs sector and is based on 
References~\cite{SusyPrimer,Djuadi}.
Finally, a review of the MSSM Higgs bosons phenomenological aspects, which are relevant to the presented search, is given 
in Section~\ref{sec:pheno} based on Reference~\cite{LHCxsec}.
%In section... introduction is given based on~\cite{Altarelli}, 
%This chapter is devoted to introduce the the Minimal Supersymmetric 
%extension of the Standard Model (MSSM) whit focus on  its Higgs sector.
%In section... introduction is given based on~\cite{Altarelli}, 

\restoregeometry

\clearpage

\section{The Standard Model of Particle Physics} \label{sec:SM}
\subsection{Introduction}
A detailed description of the Standard Model of particle physics may be found in Ref.~\cite{Peskin}, only a brief overview is
given in what follows.

The Standard Model (SM) of particle physics is a theory aimed to describe and quantitatively predict
the phenomenology of foundamental interactions. At ``microscopic''  level the spectrum of all interactions between matter and 
radiation can be understood in terms of three classes of foundamental forces: the strong, the electromagnetic
and the weak forces. These interactions are described by a local relativistic quantum field theory, where to each particle
is associated a field with suitable transformation properties under the Lorentz group.
The theory is based on the principle of gauge invariance, which means invariance under  a symmetry  transformations    
that operates on basic internal degrees of freedom and depends on the space-time coordinate.
The gravitational force is negligible in atomic and nuclear physics, in fact,
quantum effects of gravity are expected at energies corresponding to the Planck mass $E \sim M_{planck} c^2 \sim 10^{19}$~GeV.

The SM is a gauge field theory based on the symmetry group $SU(3)_c \otimes SU(2)_L \otimes U(1)_Y$. The group has $8+3+1=12$
generators with a non trivial commutator algebra. The electromagnetic and weak interactions~\cite{EW1,EW2,EW3}  are described  by the 
$SU(2)_L \otimes U(1)_Y$ symmetry group, while the $ SU(3)_c$ is the colour group of the theory of strong interactions (QCD)~\cite{qcd1}.
To each generator of the symmetry group is associated a vector boson which act as mediator of the correspondig interactions.
Eight gluons are associated to the $ SU(3)_c$ colour generators, while for four gauge bosons $W^{\pm}$,
$Z^0$ and $\gamma$ are associated to the generators of $SU(2)_L \otimes U(1)_Y$. 
Only the gluons and the photon are massless since the symmetry induced by the other three generators is
spontaneusly broken. In the SM the spontaneus symmetry breaking is realized by the Higgs mechanism~\cite{ENGLERT,HIGGS,HIGGS2,HIGGS3,kibble}.
The Higgs boson acts as mediator of a new class of interactions that, at tree level, are coupled in proportion to the particle masses.
An Higgs boson, with properties that resemble the one of the SM, 
has recently been discovered at the LHC with $m_H \sim 126$~GeV \cite{AHiggsO,CHiggsO} and represents one of the major mailstone of particle physics.

The fermionic matter fields of the SM are quarks and leptons. 
Quarks  are subject to all SM interactions, each type of quark is a colour triplet and carries 
electroweak charges, in particular electric charges $+2/3$ for up-type quarks and $-1/3$
for down-type quarks.  Leptons are colourless
but have electroweak charges, in particular electric charges $-1$ for charged leptons $e$, $\mu$ and $\tau$ (opposite sign charge 
is intended for respective anti-particle)  and charge 0 for neutrinos $\nu_e$, $\nu_{\mu}$ and $\nu_{\tau}$.
Qarks and leptons are grouped in three  ``generations'' with equal quantum numbers but different masses.


\subsection{Precision Test and Limitation of the SM}

Precision tests of the SM has been performed over a wide range of energies in experiment in the last several decades,
Precision tests~\cite{precisiontest} of the standard electroweak theory performed at LEP, SLC and Tevatron~\cite{smtest}, 
has confirmed the couplings of quark and leptons to the weak gauge bosons $W^{\pm}$  and $Z$ are indeed
precisely those prescribed by the gauge symmetry. The accuracy of a few per-mille for these
tests implies that, not only the tree level, but also the structure of quantum corrections has
been verified. Several other experimental results~\cite{pdg} including rare decays and the anomalus magnetic moment of the muon
provide a test for low-energies of the standard model. The recent discovery of a Higgs boson with mass in perfect agreement with 
the prediction of the SM~\cite{gfitter}.

In spite of this success, the Standard Model is conceptually unsatisfactory for quite few deficiencies and is
widely belived to be an effective theory valid only at the present accessible energies. Beside the fact that 
it does not  include  gravitational force, it does not explaint the pattern of fermion masses and in its simplest 
version does not include neutrino masses, it ha at least other three conceptual problems which indicates the need 
for physics Beyond the Standard Model (BSM):
\begin{itemize}
\item Calculating the radiative correction to the Higgs boson mass, quadratic divergencies of the order  
	of the cut-off scale $\Lambda$ occur, where  $\Lambda$ defines  the energy
	beyond which the theory ceases to be valid and new physics should appear~\cite{Djuadi4}. If the cut-off is choosen 
	to be $\sim M_{Planck}$, then an 
	unnatural fine tunig should occur to cancell these divergencies leaving the Higgs boson with a mass of the order of the electroweak breaking scale, $M_{EW}$.
	A question that has no satisfactory answer in the SM is how these cancellations can occur and why $\Lambda >> M_{EW}$, these problems are
	called the fine-tuning and hierarchy problem~\cite{Hierarchy1,Hierarchy2,Hierarchy3}.
	%Supersymmetry was first introduced to answer this question, indeed, the new symmetry prevents the Higgs boson mass to acquire very large 
	%radiative corrections, the quadratic divergent loop of the SM particles are exactly canceled by the loop contribution 
	%of the corresponding partners, solving in this way the hierarchy problem~\cite{Djuadi5}.

\item   The SM does not have a candidate which can explain the large contribution 
	of non-barionic, non-luminous matter to the density of the Universe~\cite{darkmatter1,darkmatter2,darkmatter3}. To be a Dark Matter candidate a particle should be
	stable, massive and should interact only via very weak interactions. 
	%Supersymmetry models,  provided with R-parity consevation,
	%provides a natural candidate for Dark Matter, which would be then the lighest SUSY particle which cannot further decay.

\item Another unsatisfactory aspect of the SM is that does not provide the unification of the electroweak and strong interactions,
	their couplings do not meet at high energies. Considering the successfull unification of electromagnetic and weak
	interaction, the existence of Grand Unified Theory (GUT) has been 
	suggested~\cite{GUT1,GUT2}, which predicts the unification of all the three gauge coupling strenght at the GUT energy scale,
	$\Lambda_{GUT} \simeq 10^{16}$~GeV and describes the three forces within a single gauge group with just one coupling constant.
	%The particle spectrum in SUSY, instead, contributes to the renormalizzation group evolution of the three gauge couplings constants and 
	%favours them to meet at energies around $10^{16}$ GeV~\cite{djuadi4}. 

\end{itemize}
Among all the extension of the SM, Supersymmetry is a theoretically favoured scenario as the most predictive framework beyond the Standard Model.
As discussed in Section~\ref{sec:MSSM}, it gives a natural answer to the hierarchy problem, provides a suitable candidate for Dark Matter
and predicts unification of the three gauge couplings at GUT energy scale.


% the conceptual situation with the
%standard model is unsatisfactory for quite a few deficiencies:
%– the smallness of the electroweak scale v ∼ 246GeV << MPl (the ‘hierarchy problem’);
%– the large number of free parameters (gauge couplings,
%vacuum expectation value, MH, fermion masses,
%CKM matrix elements), which are not predicted but have
%to be taken from experiments;
%– the pattern that occurs in the arrangement of the
%fermion masses;
%– the missing way to connect to gravity.
%
%-Dark matter
%
%-neutrino masses

 
\section{The Minimal Supersymmetric Standard Model}\label{sec:MSSM}
\subsection{Introduction to the MSSM}
Supersymmetry (SUSY)~\cite{Susy1,Susy2,Susy3} was first introduced since as a natural way to solve the hierarchy problem.
The SUSY generators $\mathcal{Q}$ transforms fermion into bosons and vice versa:
\begin{equation}
\mathcal{Q}|\text{Fermion}\rangle = |\text{Boson}\rangle, ~ ~ ~ \mathcal{Q}|\text{Boson}\rangle = |\text{Fermion}\rangle
\end{equation}
This is suggesting that in a supersymmetric extesion of the SM  each of the known foundamental particles 
is in either a chiral or gauge ``supermultiplet'' and must have a superpartner with spin differing by 1/2 unit.
In this way, the quadratic divergent loop contribution to the Higgs mass of the SM 
particles are canceled by the loop contribution of the corresponding partners. 
Since the left-handed and right-handed components of fermions transform differently under gauge transformations also their superpartner 
should maintain this property. The name of the superpartner of the quarks and leptons are made by adding an ``s'' to the SM name, standing for scalar.
Accordingly, the gauge bosons related to the generator of the group $SU(3)_c \otimes SU(2)_L \otimes U(1)_Y$ should also have a spin 1/2 partner,
whose name will be made by adding a ``ino'' at the end of the SM name. The symbol of superpartners is defined by adding a ($\tilde{ ~ }$) to the SM symbol.

The  Minimal Supersymmetric extension of the Standard Model  (MSSM)~\cite{MSSM1,MSSM2,MSSM3,MSSM4,MSSM5,MSSM6}, 
is defined by requiring the minimal gauge group (i.e., the SM one)
and the minimal particle content: three generation of fermions (without right-handed neutrinos), gauge bosons and two Higgs doublet, 
each with its superpartners. Tables~\ref{tab:chiralsup} and~\ref{tab:gaugesup} summarize chiral and gauge supermultiplets in the MSSM.
This guys will mix to form neutralino and chargino....
\begin{table}
\begin{center}
\renewcommand{\arraystretch}{1.5}
\begin{tabular}{c|ccc}
\hline%\noalign{\smallskip}
Names 			&Supermultiplets	&	Spin 1/2  		& Spin 0 \\%[0.1cm]
\hline%\noalign{\smallskip}		
quark, squarks		& $Q$ 			&	$(u_L ~ d_L)$		& $( \tilde{u}_L ~ \tilde{d}_L)$ \\%[0.1cm]
($\times$ 3 families)	& $\bar{u}$		& 	$u_R^{\dagger}$ 	& $\tilde{u}_R^*$ \\%[0.1cm]
			& $\bar{d}$		& 	$d_R^{\dagger}$ 	& $\tilde{d}_R^*$ \\%[0.1cm]
\hline%\noalign{\smallskip}
leptons, sleptons	& $L$			&   	$(\nu ~ e_L)$ 		&  $( \tilde{\nu} ~ \tilde{e}_L)$\\%[0.1cm]
($\times$ 3 families)	& $\bar{e}$		&	$e_R^{\dagger}$         & $\tilde{e}_R^*$ \\%[0.1cm]
\hline%\noalign{\smallskip}
higgsinos, Higgs	& $H_1$			&	$( \tilde{H}_1^0 ~ \tilde{H}_1^-)$  &	$( H_1^0 ~ H_1^-)$ \\%[0.1cm]	 
			& $H_1$			&	$( \tilde{H}_2^+ ~ \tilde{H}_2^0)$  &	$( H_2^+ ~ H_2^0)$ \\%[0.1cm]	 
\hline
\end{tabular}
\caption{This table is based on~\cite{SusyPrimer} and summarize the chiral supermultiplets in the Minimal Supersymmetric Standard Model. The spin-0 fields are complex scalars
	and the spin-1/2 are left-handed two-component Weyl fermions.}
\label{tab:chiralsup}
\end{center}
\end{table}

\begin{table}
\begin{center}
\renewcommand{\arraystretch}{1.5}
\begin{tabular}{c|cc}
Names			& Spin 1 		&	Spin 1/2 \\
\hline
gluon, gluino		& $g$			& $\tilde{g}$	\\
W bosons, winos		& $W^{\pm}$ $W^0$	& $\tilde{W}^{\pm}$ $\tilde{W}^0$ \\
B boson, bino		& $B^0$			& $\tilde{B}^0$ \\
\hline
\end{tabular}
\caption{This table is based on~\cite{SusyPrimer} and summarize the gauge supermultiplets in the Minimal Supersymmetric Standard Model.}
\label{tab:gaugesup}
\end{center}
\end{table}

\subsubsection{$R$-parity conservation}
The MSSM also requires a discrete and multiplicative symmetry called $R$-parity~\cite{djuadi12}, this symmetry assures barion and lepton number 
conservation and it is defined as follows:
\begin{equation}
R_p = (-1)^{2s+3B=L}
\end{equation}
where $L$ and $B$ are lepton and barion numbers and $s$ stands for the spin quantum number. R-parity quantum numbers has value $+1$ for ordinary
SM particles and $-1$ for their superpartners. This symmetry was first introduced to overcome the problem of instability of the proton,
lepton and barion number violation leads, in many cases, 
to unstable proton with life-time shorter than the experimental lower limit. However the conservation of $R$-parity has also other important 
fenomenological consequences, SUSY particle are always produced in pairs and in their decay there is always an odd number of SUSY particle, 
and the lightest SUSY particle is stable, providing a good candidate for dark matter.


\subsubsection{The Soft SUSY Breaking}
In case the Supersymmetry is an exact symmetry of nature, the bosonic fields and the corrispective fermion fields should have the same mass 
and quantum numbers, except for the spin. However, the particle spectrum of SUSY has not yet been observed, suggesting that these particle
 should have an higher mass than their SM superpartners. 
To achieve SUSY-breaking in a way which does not reintroduce the quadratic divergences to the Higgs mass squared, a so called ``soft-SUSY-breaking''
term is introduced~\cite{djuadipage5orBetterPage12}, this term explicitly break SUSY introducing ad hoc the mass terms for Higgs, gauginos and
sferions, furthermore  trilinear coupling terms between sfermions and Higgs bosons are introduced. In general, if intergenerational mixing and 
complex phases are allowed, the soft-SUSY-breaking terms will introduce a huge number of unknown parameters $\mathcal{O}(100)$~\cite{djuadiPage5}.
However, in absence of phases and  mixing, and if obey to a set of boundary conditions~\cite{djuadi5}, only few new parameters are introduced.




\subsection{The Higgs Sector in the MSSM }
In the MSSM two doublets of complex scalar field of opposite hypercharge are required to break the electroweak symmetry, 
this requirement is motivated by the needs to generate  masses separately to isospin up-type fermion and down-type fermions~\cite{djouadiP21}
and to cancel chiral anomalies that otherwise would spoil the renormalizability of the theory~\cite{djouadiP21}. The two Higgs doublet then are:
\begin{equation}
H_1 = \binom{H_1^0}{H_1^-} ~ ~ \text{with } Y_{H_1} = -1, \quad \quad H_2 = \binom{H_2^+}{H_2^0} ~ ~ \text{with } Y_{H_2} = +1  
\end{equation}
In analogy with the SM, a similar Higgs mechanism is employed in the MSSM~\cite{djuadiP22},  requiring that the minimum 
of the Higgs potential breaks $SU(2)_L \otimes U(1)_Y$ group while preserving the electromagnetic symmetry $U(1)_Q$.
the  to break electroweak symmetry. The neutral components of the 
two Higgs field acquire vacuum expectation values:
\begin{equation}
\langle H_1^0 \rangle = \frac{v_1}{\sqrt{2}}, \quad \quad \quad  \langle H_2^0 \rangle = \frac{v_2}{\sqrt{2}}
\end{equation}
Three of the original eight degrees of freedom of the scalar fields are absorbed by the $W^{\pm}$ and $Z$ bosons, building their lomgitudinal
polarizations and acquire masses. The remaning degrees of freedom correspond to five scalar Higgs bosons: two CP-even and neutral $h$ and $H$, 
a neutral pseudoscalar boson $A$ and a pair of charged bosons $H^{\pm}$. At tree level, besides the masses of these particle, two additional parameter
define the system: the mixing angle in the neutral CP-even sector $\alpha$ and the ratio between the two vacuum expectation value $\tan \beta = v_1/v_2$.
However, the supersymmetric structure of the theory impose strong constraint on the Higgs spectrum, out of the six parameters which describe 
the MSSM Higgs sector, $M_h$, $M_H$, $M_A$, $M_{H^\pm}$, $\beta$ and $\alpha$, only two  are actually independent at tree level, a common choice 
is $\tan \beta$ and $M_A$. At tree level these relation impose a strong hierarchical structure on the mass spectrum: the $h$ boson is the lightest
with  $M_h < M_Z$,  $M_A < M_H$ and $M_{H^\pm}^2 = M_A^2 M_W^2$. Furthermore, the following relation holds between the mixing angles, which is particularly important
for the Higgs couplings:
\begin{equation}
\cos^2(\beta - \alpha) = \frac{M_h^2 (M_Z^2 - M_h^2)}{M_A^2 (M_H^2 - M_h^2)}
\end{equation}
These relations, however, are broken by large radiative corrections to the Higgs 
masses~\cite{djuadiPage6}, which cause the costraint on the mass of $h$ to move from the tree level value of $M_Z$ to $\mh \apprle 140$ GeV.
Another restriction, coming from GUT assumptions gives $1 \apprle \tan \beta \apprle m_t/m_b$ ~\cite{sebpage27}.


\section{Neutral Higgs Bosons Phenomenology in the MSSM}\label{sec:pheno}

\subsection{MSSM Higgs Couplings with SM Particles}\label{sec:couplings}
The phenomenology of the MSSM Higgs bosons is enclused in their couplings with standard model and supersymmetric particles, 
a short overview of the former, based onthe review~\cite{Djuadi}, is given in this section.

The Feynman diagram for the possible couplings between MSSM Higgs bosons and vector bosons are shown in Figure~\ref{fig:couplings}, where is possible to identify threelinear 
couplings $V_{\mu}V_{\nu}H_i$ among one Higgs boson and two gauge bosons and $V_{\mu}H_{i}H_j$ among one gauge boson and two Higgs bosons,
as well as the couplings between two Higgs bosons and two gauge bosons $V_{\mu}V_{\nu}H_iH_j$.
\begin{figure}[tp]
     \begin{center}
     \subfigure[]{		
            \includegraphics[width=0.3\textwidth]{figure/blank.pdf}
     }	
     \subfigure[]{		
            \includegraphics[width=0.3\textwidth]{figure/blank.pdf}
     }	
     \subfigure[]{		
            \includegraphics[width=0.3\textwidth]{figure/blank.pdf}
     }
     \end{center}
   \label{fig:couplings}
    \caption{Feynman diagrams for the couplings between one Higgs boson and two gauge bosons (a), two Higgs bosons and one gauge boson (b)
		and two Higgs bosons and two gauge bosons (c). Based on~\cite{Djuadi}. }
\end{figure}
Among these couplings, the most relevat for MSSM Higgs phenomenology is the trilinear couplings between two gauge bosons and one Higgs boson $V_{\mu}V_{\nu}H_i$,
in this case, since the photon is massless, there are no Higgs-$\gamma\gamma$ and Higgs-$Z\gamma$ couplings at tree level, CP-invariance also forbids $WWA$, $ZZA$
and $WZH^{\pm}$ couplings, for this case only the following couplings remains:
\begin{align} \label{eq:couplings}
Z_{\mu}Z_{\nu} h ~ :  ~ & ig_z M_Z \sin(\beta -\alpha) g_{\mu\nu},  &  Z_{\mu}Z_{\nu} H ~ : ~  ~    & ig_z M_Z \cos(\beta -\alpha) g_{\mu\nu} \\
W_{\mu}^+W_{\nu}^- h ~: ~&  ig_w M_W \sin(\beta -\alpha) g_{\mu\nu},  &  W_{\mu}^+W_{\nu}^- H ~ : ~ ~ & ig_w M_W \cos(\beta -\alpha) g_{\mu\nu}
\end{align}
The couplings of the neutral CP-even Higgs bosons $h$ and $H$ with pair of vector bosons are prortional to $ \sin(\beta -\alpha)$ and $\cos(\beta -\alpha)$
respectively, where $\cos(\beta -\alpha)$ is fixed at tree level following equation~\eqref{eq:couplings}. An interesting fenomenological consequence is
that, calling $G_{VVh}$ and $G_{VVH}$ a general   coupling between two vector bosons and one of the neutral CP-even Higgs bosons the following equation holds:
\begin{equation}\label{eq:couplingSM}
G^2_{VVh} +G^2_{VVH} = g^2_{VVH_{SM}}
\end{equation}
this means that the couplings with vector bosons for $h$ and $H$ respectively increase and decrease with $\tan\beta$, for
large value of $\tan\beta$, $h$ has SM-like couplings with vector bosons and $H$  decouple from them. For an overview of all the other
couplinq between vector boson and Higgs bosons, charged Higgs, trilinear and quartic coupling between Higgs bosons and couplings 
to SUSY particles refer to~\cite{Djuadi}.

The MSSM Higgs bosons couplings with isospin up-type $u$, and down-type $d$ fermions also depend on $\tan\beta$ and may be written
as follows:
\begin{small}
\begin{align*}
G_{huu} ~\propto ~ & m_u [\sin(\beta - \alpha)  + \cot\beta \cos(\beta - \alpha)], & G_{hdd} ~\propto ~ & m_u [\sin(\beta - \alpha)  - \tan\beta \cos(\beta - \alpha)]\\
G_{Huu} ~\propto ~& m_u [\cos(\beta - \alpha)  - \cot\beta \sin(\beta - \alpha)], & G_{Hdd} ~\propto~  & m_d [\cos(\beta - \alpha)  + \tan\beta \sin(\beta - \alpha)]\\
G_{Auu} ~ \propto ~ & m_u  \cot\beta, & G_{Add} ~ \propto ~ & m_d \tan\beta 
\end{align*} 
\end{small}
Then the couplings with down-type (up-type) fermions of either the $h$ or $H$ boson is enhanced (suppressed) by a factor $\tan\beta$, depending
on the magnitude of $\cos(\beta - \alpha)$ or $\sin(\beta - \alpha)$, while the coupling of $A$ boson with down-type (up-type) fermions are directly 
enhanced (suppressed) by $\tan\beta$.


\subsection{MSSM Higgs Benchmark Scenarios}
In the MSSM  the Higgs sector is fully determined at tree level by two free parameter by convention chosen to be $M_A$ and $\tan\beta$, 
meaning that the Higgs bosons masses, decay branching fraction and production cross section are all determined. However, radiative corrections
contribute significantly to the Higgs masses~\cite{sebThesis26}, these physical quatintity becomes dependent on several other parameters.
The main corrections arises from the top-stop quark sector, and fro large $\tan\beta$ also from the bottom-sbottom quark sector,
and in general dependent on the SUSY-breacking scale $M_{SUSY}$, the trilinear Higgs Yukawa couplings.

Due to the large number of free parameters, a complete scan of the MSSM parameter space is impractical in experimental analysis and phenomenological
studies. Several benchmark scenarios has been proposed~\cite{LHCxsec}, where the SUSY parameters, entering via radiative corrections, 
are fixed to particular benchmark values which exibit interesting features of the MSSM Higgs phenomenology, while the parameters 
$M_A$ and $\tan\beta$ are left free to vary, and usually results are presented in a $M_A-\tan\beta$ plane.

One of the most interesting benchmark scenarios is the $m_h^{max}$ benchmark scenario~\cite{seb26}, where the parameter that contributes to radiative
corrections are fixed such that the mass of the light CP-even Higgs boson, $M_h$, is maximal under the variation of $M_A$ and $\tan\beta$. This
scenario allows to set conservative lower bounds on $M_A$, $M_H^{\pm}$ and $\tan\beta$~\cite{Xsec243}. This scenario was used in the past 
to set limits at LEP, Tevatron and LHC~\cite{LEPLimits,TevatronLimits1,CMSLimit,ATLASLimit}, 
however, given the recent discovery of a Higgs boson with mass \~125.5 GeV, this scenario
tend to predict a too high mass for $M_h$, resulting to be, for large part of the parameter space, inconsistent with this observation
by roughly two $\sigma$. The mass of $M_h$ is bounded to be $M_h < 135$ GeV.

An interesting up-to-date benchmark scenario is the $m_h^{mod\pm}$ benchmark scenario, where departing from the parameter configuration of the $m_h^{max}$
the amount of mixing in the stop sector is reduced and has the feature of predict that the lightest CP-even Higgs boson should have a mass
$M_h \simeq 125.5 \pm 3 $ GeV. The low-$M_H$ benchmark scenario is complementary to the  $m_h^{mod\pm}$, in this scenario, the observed 
Higgs boson is instead interpreted as the heavy CP-even Higgs boson of the MSSM, implying that $h$ and $A$ should have relatively small mass.
For an overview of the interesting benchmark scenario refer to~\cite{LHCxsec}. 


The $m_h^{max}$ benchmark scenario has been used throughout this thesis, since it provides a model that can be compared with past exclusion limits
and it has been shown that experimentally, the resolution to the Higgs mass in the $\tau\tau$ decay mode is not enough to distinguish between
the $m_h^{max}$ scenario and an updated  scenario in which $M_h \simeq 125.5 \pm 3 $ GeV for all the points in the parameter space. More over
it is a ``general purpose'' scenario in which exclusion limits can be assesed from low to very large value of $M_A$, which is not the case
otherwise due to the Higgs masses hierarchy.


 


\subsection{Neutral MSSM Higgs Bosons Production and Decay at LHC}
For large region of the MSSM parameter space a SM-like Higgs boson is expected, usually identified with the lightest 
CP-even Higgs boson, $h$. 
Given the Higgs bosons couplings discussed in Section~\ref{sec:couplings} tourns out that, in this case, $H$ and $A$, 
tend to be degenerate in mass and decouple from gauge bosons. Furthermore the coupling of the latter
two Higgses with down (up) type fermions are enhanced (suppressed) by $\tan\beta$, therefore, for large $\tan\beta$
bottom-quark and $\tau$ lepton will play a more important role than in
the SM case either for production and decay. 

\begin{figure}[tp]
     \begin{center}

            \includegraphics[width=0.5\textwidth]{figure/br.png}

    \end{center}
    \caption{Branching fraction for the MSSM neutral higgses $h/H/A$ in the $m_h^{max}$ scenario.}
   \label{fig:br}

\end{figure}

The production of the neutral $CP$-even MSSM Higgs bosons at hadron
colliders proceeds via the same processes as for the SM Higgs
production. However, the pseudoscalar $A$ instead cannot be produced
in association with gauge bosons or in vector boson fusion (VBF) at
tree-level, as this coupling is forbidden due to $CP$-invariance.  At
the LHC one of the most relevant production mechanisms for the MSSM
Higgs bosons is gluon-gluon fusion, $gg\rightarrow A/H/h$. In
addition, the production in association with $b$-quarks becomes
important for large value of $\tan\beta$. Those are the two production mechanism
that are considered in this analysis, Figure~\ref{fig:prod} shows the Feynman-diagram
for those processes, while Figure~\ref{fig:xsec} shows the production cross section of the neutral 
MSSM Higgses via these two processes in the $m_h^{max}$ scenario.


The decays of the neutral
MSSM Higgs bosons (in the assumption that all supersymmetric particle
are heavy enough) are the same as for the SM one with the already
cited exception of $A$. Figure~\ref{fig:xsec} shows the decay branching fractions
for $H$ and $A$ as a function of the mass, 
the decay into tau pair is the most important after $b\bar{b}$ and the one used in this thesis. 


\begin{figure}[tp]
     \begin{center}

            \includegraphics[height=3cm]{figure/blank.pdf}
            \includegraphics[height=3cm]{figure/blank.pdf}

    \end{center}
    \caption{Feynman diagram for b-associated production and gluon-gluon fusion for MSSM neutral Higgs.}
   \label{fig:prod}
\end{figure}

\begin{figure}[tp]
     \begin{center}

            \includegraphics[width=\textwidth]{figure/xsec.png}

    \end{center}
    \caption{Production cross section for the \emph{h/H/A} MSSM neutral Higgs bosons via b-associated production and
	gluon-gluon fusion production mode. The calculation are for the $m_h^{max}$ scenario and for $\tan \beta=5$ (left) and $\tan \beta=30$ (right).}
   \label{fig:xsec}
\end{figure}




\subsection{Current Status of the Seacrh for Neutral MSSM Higgs Bosons}

The measure of the couplings of the found SM-like Higgs boson can shed light on the Higgs sector and determine if this boson
is fully responsable for the generation of all the SM particles masses. 
%In fact, the couplings are sensitive to new physics,
%given the unitarity property of scattering aplitudes for longitudinal vectors and fermions, 
There are then two approaches to explore the Higgs sector: one, is to use the measured Higgs couplings with SM particles to 
set constraint on new physics, while the other is to directly search for additional Higgses in a well defined model, like in this case the MSSM.

In case the SM-like Higgs boson is interpreted as the light CP-even Higgs boson of the MSSM, the couplings of the Higgs boson 
to vector bosons ($k_V$), up-type fermions ($k_u$) and down-type fermions ($k_d$), can be expressed as a ratio to the corresponding SM expectation
and this allow to set exclusion limits in the $m_A - \tan\beta$ plane~\cite{AtlasConstraint}. Figure~\ref{fig:ex1} shows the exclusion limits in a 
``simplified MSSM'' model~\cite{atlasCostraint11} via fits to the measured rates of Higgs boson production and decay.

 
\begin{figure}[tp]
     \begin{center}

            \includegraphics[height=5cm]{figure/limits/constraintAtlas.pdf}

    \end{center}
    \caption{Regions of the  $m_A - \tan\beta$ plane excluded in a simplified MSSM model via fits to the measured
%rates of Higgs boson production and decays. The likelihood contours where $−2 \ln \Lambda = 6.0$, corresponding
approximately to 95\% CL ($2\sigma$), are indicated for the data and expectation assuming the SM Higgs sector.
The light shaded and hashed regions indicate the observed and expected exclusions, respectively. The
SM decoupling limit is $m_A \rightarrow \infty$. See Reference~\cite{AtlasConstraint}.}

   \label{fig:ex1}
\end{figure}


The current latest constraint on $m_A - \tan\beta$  by direct search of neutral MSSM Higgs bosons are instead shown in Figure~\ref{}
and are part of the work of this thesis.






















