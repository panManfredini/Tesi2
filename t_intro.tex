\newgeometry{left=4.5cm, right=4.5cm,bottom=4cm, top=4cm}

\chapter{The Higgs Bosons and the MSSM}

 \vspace{2cm}

This chapter is devoted to introduce the theoretical background to the experimental search presented in this thesis.
A brief overview of the Standard Model of particle physisics is given in Section~\ref{sec:SM} based on Reference~\cite{Altarelli}. 
Among all the extesion of the Standard Model, the Minimal Supersymmetric 
extension of the Standard Model (MSSM) is a theoretically favoured scenario as one of the most predictive
framework beyond the Standard Model, it is introduced in Section~\ref{sec:MSSM} with focus on its Higgs sector and is based on 
References~\cite{SusyPrimer,Djuadi}.
Finally, a review of the MSSM Higgs bosons phenomenological aspects, which are relevant to the presented search, is given 
in Section~\ref{sec:pheno} based on Reference~\cite{LHCxsec}.
%In section... introduction is given based on~\cite{Altarelli}, 
%This chapter is devoted to introduce the the Minimal Supersymmetric 
%extension of the Standard Model (MSSM) whit focus on  its Higgs sector.
%In section... introduction is given based on~\cite{Altarelli}, 

\restoregeometry

\clearpage

\section{The Standard Model of Particle Physics} \label{sec:SM}
\subsection{Introduction}
A detailed description of the Standard Model of particle physics may be found in Ref.~\cite{Peskin}, only a brief overview is
given in what follows.

The Standard Model (SM) of particle physics is a theory aimed to describe and quantitatively predict
the phenomenology of foundamental interactions. At ``microscopic''  level the spectrum of all interactions between matter and 
radiation can be understood in terms of three classes of foundamental forces: the strong, the electromagnetic
and the weak forces. These interactions are described by a local relativistic quantum field theory, where to each particle
is associated a field with suitable transformation properties under the Lorentz group.
The theory is based on the principle of gauge invariance, which means invariance under  a symmetry  transformations    
that operates on basic internal degrees of freedom and depends on the space-time coordinate.
The gravitational force is negligible in atomic and nuclear physics, in fact,
quantum effects of gravity are expected at energies corresponding to the Planck mass $E \sim M_{planck} c^2 \sim 10^{19}$~GeV.

The SM is a gauge field theory based on the symmetry group $SU(3)_c \otimes SU(2)_L \otimes U(1)_Y$. The group has $8+3+1=12$
generators with a non trivial commutator algebra. The electromagnetic and weak interactions~\cite{EW1,EW2,EW3}  are described  by the 
$SU(2)_L \otimes U(1)_Y$ symmetry group, while the $ SU(3)_c$ is the colour group of the theory of strong interactions (QCD)~\cite{qcd1}.
To each generator of the symmetry group is associated a vector boson which act as mediator of the correspondig interactions.
Eight gluons are associated to the $ SU(3)_c$ colour generators, while for four gauge bosons $W^{\pm}$,
$Z^0$ and $\gamma$ are associated to the generators of $SU(2)_L \otimes U(1)_Y$. 
Only the gluons and the photon are massless since the symmetry induced by the other three generators is
spontaneusly broken. In the SM the spontaneus symmetry breaking is realized by the Higgs mechanism~\cite{ENGLERT,HIGGS,HIGGS2,HIGGS3,kibble}.
The Higgs boson acts as mediator of a new class of interactions that, at tree level, are coupled in proportion to the particle masses.
An Higgs boson, with properties that resemble the one of the SM, 
has recently been discovered at the LHC with $m_H \sim 126$~GeV \cite{AHiggsO,CHiggsO} and represents one of the major mailstone of particle physics.

The fermionic matter fields of the SM are quarks and leptons. 
Quarks  are subject to all SM interactions, each type of quark is a colour triplet and carries 
electroweak charges, in particular electric charges $+2/3$ for up-type quarks and $-1/3$
for down-type quarks.  Leptons are colourless
but have electroweak charges, in particular electric charges $-1$ for charged leptons $e$, $\mu$ and $\tau$ (opposite sign charge 
is intended for respective anti-particle)  and charge 0 for neutrinos $\nu_e$, $\nu_{\mu}$ and $\nu_{\tau}$.
Qarks and leptons are grouped in three  ``generations'' with equal quantum numbers but different masses.


\subsection{Precision Test and Limitation of the SM}

The Standard Model has been succesflully tested in a vast number of experiments over a wide range of energies during the last few decades.
Precision tests of the electroweak theory performed at LEP, SLC and the Tevatron~\cite{smtest}, 
has confirmed that the couplings of quark and leptons to the weak gauge bosons $W^{\pm}$  and $Z$ are indeed
precisely those prescribed by the gauge symmetry. The accuracy of a few per-mille for these
tests implies that, not only the tree level, but also the structure of quantum corrections has
been verified. Several other experimental results~\cite{pdg}
including rare decays of hadrons provides a test for low-energies of the standard model. 
The recent discovery of a Higgs boson  is also another success of the SM, its mass, spin and couplings 
are in agreement with expected values from a global fit of electroweak costraints~\cite{gfitter}.
Among all the parameters of the Standard Model only few of them presents tension with experimental data, 
the most significant discrepancies are sligthly above three standard deviations and are given by: the anomalus magnetic moment 
of the muon~\cite{gminus2}, $a_{\mu}$  and the forward-backward asymmetry of top quarks~\cite{FBasymmetry}, $A_{FB}^{\ttbar}$.   


In spite of this success, the Standard Model is conceptually unsatisfactory for quite few deficiencies and is
widely belived to be an effective theory valid only at the present accessible energies. Beside the fact that 
it does not  include  gravitational force, it does not explain the pattern of fermion masses and in its simplest 
version does not include neutrino masses, it has at least other three conceptual problems which indicates the need 
for physics Beyond the Standard Model (BSM):
%\begin{itemize}
%\item
\paragraph{Hierarchy Problem} Calculating the radiative correction to the Higgs boson mass, quadratic divergencies of the order  
	of the cut-off scale $\Lambda$ occur, where  $\Lambda$ defines  the energy
	beyond which the theory ceases to be valid and new physics should appear~\cite{Lambda}. If the cut-off is choosen 
	to be $\sim M_{Planck}$, then a fine tunig  with an unnaturally high precision , $\mathcal{O}(10^{-30})$, should occur to cancell these divergencies 
	leaving the Higgs boson with a mass of the order of the electroweak breaking scale, $M_{EW}$.
	A question that has no satisfactory answer in the SM is how these cancellations can occur and why $\Lambda >> M_{EW}$, these problems are
	called the fine-tuning and hierarchy problem~\cite{Hierarchy1,Hierarchy2,Hierarchy3}.
	%Supersymmetry was first introduced to answer this question, indeed, the new symmetry prevents the Higgs boson mass to acquire very large 
	%radiative corrections, the quadratic divergent loop of the SM particles are exactly canceled by the loop contribution 
	%of the corresponding partners, solving in this way the hierarchy problem~\cite{Djuadi5}.

%\item
\paragraph{Dark Matter}   The SM does not have a candidate which can explain the large contribution 
	of non-barionic, non-luminous matter to the density of the Universe~\cite{darkmatter1,darkmatter2,darkmatter3}. To be a Dark Matter candidate a particle should be
	stable, massive and should interact only via very weak interactions. 
	%Supersymmetry models,  provided with R-parity consevation,
	%provides a natural candidate for Dark Matter, which would be then the lighest SUSY particle which cannot further decay.

%\item 
\paragraph{Unification Problem}	Another unsatisfactory aspect of the SM is that does not provide the unification of the electroweak and strong interactions,
	their couplings do not meet at high energies. Considering the successfull unification of electromagnetic and weak
	interaction, the existence of Grand Unified Theory (GUT) has been 
	suggested~\cite{GUT1,GUT2}, which predicts the unification of all the three gauge coupling strenght at the GUT energy scale,
	$\Lambda_{GUT} \simeq 10^{16}$~GeV and describes the three forces within a single gauge group with just one coupling constant.
	%The particle spectrum in SUSY, instead, contributes to the renormalizzation group evolution of the three gauge couplings constants and 
	%favours them to meet at energies around $10^{16}$ GeV~\cite{djuadi4}. 

%\end{itemize}
Among all the extension of the SM, Supersymmetry is a theoretically favoured scenario as the most predictive framework beyond the Standard Model.
As discussed in Section~\ref{sec:MSSM}, it gives a natural answer to the hierarchy problem, provides a suitable candidate for Dark Matter
and predicts unification of the three gauge couplings at GUT energy scale.


% the conceptual situation with the
%standard model is unsatisfactory for quite a few deficiencies:
%– the smallness of the electroweak scale v ∼ 246GeV << MPl (the ‘hierarchy problem’);
%– the large number of free parameters (gauge couplings,
%vacuum expectation value, MH, fermion masses,
%CKM matrix elements), which are not predicted but have
%to be taken from experiments;
%– the pattern that occurs in the arrangement of the
%fermion masses;
%– the missing way to connect to gravity.
%
%-Dark matter
%
%-neutrino masses

 
\section{The Minimal Supersymmetric Standard Model}\label{sec:MSSM}
\subsection{Introduction to the MSSM}
Supersymmetry (SUSY)~\cite{Susy1,Susy2,Susy3} was first introduced as a natural way to solve the hierarchy problem.
The SUSY generators $\mathcal{Q}$ transforms fermion into bosons and vice versa:
\begin{equation}
\mathcal{Q}|\text{Fermion}\rangle = |\text{Boson}\rangle, ~ ~ ~ \mathcal{Q}|\text{Boson}\rangle = |\text{Fermion}\rangle
\end{equation}
In a supersymmetric extesion of the SM  each of the known foundamental particles 
is in either a chiral or gauge \emph{supermultiplet} and must have a superpartner with spin differing by 1/2 unit.
SUSY naturally solve the hierarchy problem since the quadratic divergent loop contribution to the Higgs mass of the SM 
particles are canceled by the loop contribution of the corresponding partners. 
The name of the superpartner of the quarks and leptons are made by adding an ``s'' to the SM name, standing for scalar.
Accordingly, the gauge bosons related to the generator of the group $SU(3)_c \otimes SU(2)_L \otimes U(1)_Y$ should also have a spin 1/2 partner,
whose name will be made by adding a ``ino'' at the end of the SM name. The symbol of superpartners is defined by adding a ($\tilde{ ~ }$) to the SM symbol.
The SUSY particles share the same couplings with their SM partner, since the left-handed and right-handed components of fermions 
transform differently under gauge transformations also their superpartner present this feature.

The  Minimal Supersymmetric extension of the Standard Model  (MSSM)~\cite{MSSM1,MSSM2,MSSM3,MSSM4,MSSM5,MSSM6}, 
is defined by requiring the minimal gauge group (i.e., the SM one)
and the minimal particle content: three generation of fermions (without right-handed neutrinos), gauge bosons and two Higgs doublet, 
each with its superpartner. Tables~\ref{tab:chiralsup} and~\ref{tab:gaugesup} summarize chiral and gauge supermultiplets in the MSSM.
Among the gauge eigenstates summarized in the tables, the superpartner of the Higgs bosons, the \emph{higgsinos} mix with the \emph{wino}
and \emph{bino} to give the ``ino'' mass eigenstates: two charginos $\chi_{1,2}^\pm$ and four neutralinos $\chi_{1,2,3,4}^0$.
\begin{table}
\begin{center}
\renewcommand{\arraystretch}{1.5}
\begin{tabular}{c|ccc}
\hline%\noalign{\smallskip}
Names 			&Supermultiplets	&	Spin 1/2  		& Spin 0 \\%[0.1cm]
\hline%\noalign{\smallskip}		
quark, squarks		& $Q$ 			&	$(u_L ~ d_L)$		& $( \tilde{u}_L ~ \tilde{d}_L)$ \\%[0.1cm]
($\times$ 3 families)	& $\bar{u}$		& 	$u_R^{\dagger}$ 	& $\tilde{u}_R^*$ \\%[0.1cm]
			& $\bar{d}$		& 	$d_R^{\dagger}$ 	& $\tilde{d}_R^*$ \\%[0.1cm]
\hline%\noalign{\smallskip}
leptons, sleptons	& $L$			&   	$(\nu ~ e_L)$ 		&  $( \tilde{\nu} ~ \tilde{e}_L)$\\%[0.1cm]
($\times$ 3 families)	& $\bar{e}$		&	$e_R^{\dagger}$         & $\tilde{e}_R^*$ \\%[0.1cm]
\hline%\noalign{\smallskip}
higgsinos, Higgs	& $H_1$			&	$( \tilde{H}_1^0 ~ \tilde{H}_1^-)$  &	$( H_1^0 ~ H_1^-)$ \\%[0.1cm]	 
			& $H_1$			&	$( \tilde{H}_2^+ ~ \tilde{H}_2^0)$  &	$( H_2^+ ~ H_2^0)$ \\%[0.1cm]	 
\hline
\end{tabular}
\caption{This table is based on Ref.~\cite{SusyPrimer} and summarize the chiral supermultiplets in the Minimal Supersymmetric Standard Model. The spin-0 fields are complex scalars
	and the spin-1/2 are left-handed two-component Weyl fermions.}
\label{tab:chiralsup}
\end{center}
\end{table}

\begin{table}
\begin{center}
\renewcommand{\arraystretch}{1.5}
\begin{tabular}{c|ccc}
Names			&Supermultiplets& Spin 1 		&	Spin 1/2 \\
\hline
gluons, gluinos		&$G_a$ (a =1,...,8)	& $g$			& $\tilde{g}$	\\
W bosons, winos		& $W_a$ (a=1,...,3)	& $W^{\pm}$ $W^0$	& $\tilde{W}^{\pm}$ $\tilde{W}^0$ \\
B boson, bino		&$B$			& $B^0$			& $\tilde{B}^0$ \\
\hline
\end{tabular}
\caption{This table is based on Ref.~\cite{SusyPrimer} and summarize the gauge supermultiplets in the Minimal Supersymmetric Standard Model.}
\label{tab:gaugesup}
\end{center}
\end{table}

\subsubsection{$R$-parity conservation}
The MSSM requires a discrete and multiplicative symmetry called $R$-parity~\cite{Susy3}, this symmetry assures barion and lepton number 
conservation and it is defined as follows:
\begin{equation}
R_p = (-1)^{2s+3B=L}
\end{equation}
where $L$ and $B$ are lepton and barion numbers and $s$ stands for the spin quantum number. The R-parity quantum number has value $+1$ for ordinary
SM particles and $-1$ for their superpartners. This symmetry was first introduced as a simple way to overcome the problem of instability of the proton,
lepton and barion number violation leads, in many cases, 
to unstable proton with life-time shorter than the experimental lower limit. The conservation of $R$-parity has also other important 
fenomenological consequences: SUSY particles are always produced in pairs and decays always in an odd number of SUSY particles.
Furthermore, the lightest SUSY particle, often chosen as one of the neutralinos, is stable and provides a suitable 
candidate for dark matter.


\subsubsection{The Soft SUSY Breaking}
In case  Supersymmetry is an exact symmetry of nature, the SM particles and their relative superpartners should have the same mass 
and quantum numbers, except for the spin. However, the particle spectrum of SUSY has not yet been observed, suggesting that,
if  these particles exist, they  should have an higher mass than their SM superpartners. 
To achieve SUSY-breaking in a way which does not reintroduce the quadratic divergences to the Higgs mass squared, a so called ``soft-SUSY-breaking''
term is introduced to the SUSY lagrangian~\cite{softerm1,softerm2}, this term explicitly break SUSY introducing the mass terms for Higgs, gauginos and
sfermions, as well as  trilinear coupling terms between sfermions and Higgs bosons. In general, if intergenerational mixing and 
complex phases are allowed, the soft-SUSY-breaking terms will introduce a huge number of unknown parameters $\mathcal{O}(100)$~\cite{softerm3}.
However, in absence of phases and  mixing, and if the soft terms obey  a set of boundary conditions~\cite{softerm1,softerm2}, 
only few new parameters are introduced $\mathcal{O}(10)$.




\subsection{The Higgs Sector in the MSSM }\label{sec:hsector}
In the MSSM two doublets of complex scalar field of opposite hypercharge are required to break the electroweak symmetry, 
this requirement is necessary to generate masses separately for isospin up-type fermion and down-type fermions~\cite{Susy2,Higgsm1,Higgsm2}
and to cancel chiral anomalies that otherwise would spoil the renormalizability of the theory~\cite{Higgsm3}. The two Higgs doublet  are:
\begin{equation}
H_1 = \binom{H_1^0}{H_1^-} ~ ~ \text{with } Y_{H_1} = -1, \quad \quad H_2 = \binom{H_2^+}{H_2^0} ~ ~ \text{with } Y_{H_2} = +1  
\end{equation}
In analogy with the SM, a similar Higgs mechanism is employed in the MSSM~\cite{MSSM1,Higgsm4}  requiring that the minimum 
of the Higgs potential breaks $SU(2)_L \otimes U(1)_Y$ group while preserving the electromagnetic symmetry $U(1)_Q$.
The neutral components of the 
two Higgs field acquire vacuum expectation values:
\begin{equation}
\langle H_1^0 \rangle = \frac{v_1}{\sqrt{2}}, \quad \quad \quad  \langle H_2^0 \rangle = \frac{v_2}{\sqrt{2}}
\end{equation}
Three of the original eight degrees of freedom of the scalar fields are absorbed by the $W^{\pm}$ and $Z$ bosons, which acquire
their longitudinal polarizations and masses. The remaning degrees of freedom correspond to five scalar Higgs bosons: two CP-even and neutral $h$ and $H$, 
a neutral CP-odd boson $A$ and a pair of charged bosons $H^{\pm}$. Six parameters describes the MSSM Higgs sector: $M_h$, $M_H$, $M_A$, $M_{H^\pm}$,
$\beta$ and $\alpha$, where the latter represents the mixing angle in the neutral CP-even sector, while $\tan \beta $ is equal to the 
ratio between the two vacuum expectation values $\tan \beta = v_1/v_2$.
At tree level,  only two of these parameters  are actually independent, a common choice is to keep $\tan \beta$ and $M_A$ as free the parameters of the Higgs sector. 
At tree level, the supersymmetric structure of the theory impose a strong hierarchical structure on the Higgs bosons mass spectrum: 
the $h$ boson is the lightest with  $M_h < M_Z$, while $M_A < M_H$ and $M_{H^\pm}^2 = M_A^2 M_W^2$. Furthermore, the 
following relation holds between the mixing angles:
\begin{equation}\label{eq:mixing}
\cos^2(\beta - \alpha) = \frac{M_h^2 (M_Z^2 - M_h^2)}{M_A^2 (M_H^2 - M_h^2)}
\end{equation}
These relations are broken by large radiative corrections to the Higgs bosons 
masses~\cite{Higgsm5}, which cause the costraint on the mass of $h$ to move from the tree level value of $M_Z$ to $\mh \apprle 140$ GeV.
Another restriction, coming from GUT assumptions gives $1 \apprle \tan \beta \apprle m_t/m_b$ ~\cite{Higgsm6}.


\section{Neutral Higgs Bosons Phenomenology in the MSSM}\label{sec:pheno}

\subsection{MSSM Higgs Couplings with SM Particles}\label{sec:couplings}
The phenomenology of the MSSM Higgs bosons depends on their couplings with standard model and supersymmetric particles, 
a short overview of the former, based on the Ref.~\cite{Djuadi}, is given in this section.

%The Feynman diagram for 
The possible couplings between MSSM Higgs bosons and vector bosons 
%are shown in Figure~\ref{fig:couplings}, where is possible to identify 
are: the threelinear couplings $V_{\mu}V_{\nu}H_i$ among one Higgs boson and two gauge bosons and $V_{\mu}H_{i}H_j$ among one gauge boson and two Higgs bosons,
as well as the couplings between two Higgs bosons and two gauge bosons $V_{\mu}V_{\nu}H_iH_j$. Figure~\ref{fig:couplings} shows the Feynman diagram 
relative to these couplings.
\begin{figure}[tp]
     \begin{center}
     \subfigure[]{		
            \includegraphics[height=3cm]{feyn_diagrams/diagrams/couplingHVV.pdf}
     }	\hspace{0.5cm}
     \subfigure[]{		
            \includegraphics[height=3cm]{feyn_diagrams/diagrams/couplingVHH.pdf}
     }	\hspace{0.5cm}
     \subfigure[]{		
            \includegraphics[height=3cm]{feyn_diagrams/diagrams/couplingVVHH.pdf}
     }
     \end{center}
   \label{fig:couplings}
    \caption{Feynman diagrams for the couplings between one Higgs boson and two gauge bosons (a), two Higgs bosons and one gauge boson (b)
		and two Higgs bosons and two gauge bosons (c). Based on~\cite{Djuadi}. }
\end{figure}
Among all of them, the most relevat for MSSM Higgs phenomenology is the trilinear couplings between two gauge bosons and one Higgs boson $V_{\mu}V_{\nu}H_i$.
For this case, since the photon is massless, there are no Higgs-$\gamma\gamma$ and Higgs-$Z\gamma$ couplings at tree level, CP-invariance also forbids $WWA$, $ZZA$
and $WZH^{\pm}$ couplings. Then, in case of $V_{\mu}V_{\nu}H_i$ couplings, only the following possibilities remains:
\begin{align} 
Z_{\mu}Z_{\nu} h ~ :  ~ & ig_z M_Z \sin(\beta -\alpha) g_{\mu\nu},  &  Z_{\mu}Z_{\nu} H ~ : ~  ~    & ig_z M_Z \cos(\beta -\alpha) g_{\mu\nu} \\
W_{\mu}^+W_{\nu}^- h ~: ~&  ig_w M_W \sin(\beta -\alpha) g_{\mu\nu},  &  W_{\mu}^+W_{\nu}^- H ~ : ~ ~ & ig_w M_W \cos(\beta -\alpha) g_{\mu\nu} \label{eq:couplings} 
\end{align}
The couplings of the neutral CP-even Higgs bosons $h$ and $H$ with pair of vector bosons are prortional to $ \sin(\beta -\alpha)$ and $\cos(\beta -\alpha)$
respectively, where $\cos(\beta -\alpha)$ is fixed at tree level following equation~\eqref{eq:mixing}. An interesting fenomenological consequence is
that, calling $G_{VVh}$ and $G_{VVH}$ the coupling between two generic vector bosons and one of the neutral CP-even Higgs bosons the following equation holds:
\begin{equation}\label{eq:couplingSM}
G^2_{VVh} +G^2_{VVH} = g^2_{VVH_{SM}}
\end{equation}
The equations~\eqref{eq:couplings} and~\eqref{eq:couplingSM} leads to the fact that the couplings with vector bosons for $h$ ($H$) 
increase (decrease) with $\tan\beta$. For relatively large value of $\tan\beta$, $h$ has SM-like couplings with vector bosons while
 $H$  decouple from them. For an overview of all the other
couplings between vector bosons and Higgs bosons, charged Higgs, trilinear and quartic coupling between Higgs bosons and couplings 
to SUSY particles refer to Ref.~\cite{Djuadi}.

The MSSM Higgs bosons couplings with isospin up-type $u$, and down-type $d$ fermions also depend on $\tan\beta$ and may be written
as follows:
\begin{small}
\begin{align*}
G_{huu} ~\propto ~ & m_u [\sin(\beta - \alpha)  + \cot\beta \cos(\beta - \alpha)], & G_{hdd} ~\propto ~ & m_u [\sin(\beta - \alpha)  - \tan\beta \cos(\beta - \alpha)]\\
G_{Huu} ~\propto ~& m_u [\cos(\beta - \alpha)  - \cot\beta \sin(\beta - \alpha)], & G_{Hdd} ~\propto~  & m_d [\cos(\beta - \alpha)  + \tan\beta \sin(\beta - \alpha)]\\
G_{Auu} ~ \propto ~ & m_u  \cot\beta, & G_{Add} ~ \propto ~ & m_d \tan\beta 
\end{align*} 
\end{small}
The couplings with down-type (up-type) fermions of either the $h$ or $H$ boson is enhanced (suppressed) by a factor $\tan\beta$, depending
on the magnitude of $\cos(\beta - \alpha)$ or $\sin(\beta - \alpha)$, while the coupling of $A$ boson with down-type (up-type) fermions are directly 
enhanced (suppressed) by $\tan\beta$.


\subsection{MSSM Higgs Benchmark Scenarios}
At tree level, the MSSM  Higgs bosons masses, decay branching fraction and production cross section are all determined by two parameters,
by convention chosen to be $M_A$ and $\tan\beta$. As it has been pointed out in Section~\ref{sec:hsector}, radiative corrections
contribute significantly to the MSSM Higgs bosons masses and the prediction of physics observables becomes 
dependent on several MSSM parameters.
The main corrections arises from the top-stop (s)quark sector and for large $\tan\beta$ also the bottom-sbottom (s)quark sector becomes increasingly 
important. Furthermore, these corrections are dependent on the SUSY-breaking scale $M_{SUSY}$, the trilinear Higgs-stop, 
Higgs-sbottom Yukawa couplings, the electroweak gaugino and gluino mass parameters.

Due to the large number of free parameters, a complete scan of the MSSM parameter space is impractical in experimental analysis and phenomenological
studies. To cope with this difficulty several benchmark scenarios has been proposed~\cite{LHCxsec,mhmax2}, where the SUSY parameters
 entering via radiative corrections 
are fixed to particular benchmark values which exibit interesting features of the MSSM Higgs phenomenology, while the parameters 
$M_A$ and $\tan\beta$ are left free to vary. Usually results are presented in a $M_A-\tan\beta$ plane.

The $m_h^{max}$ benchmark scenario~\cite{MSSMmhmax} was  used in the past searches for neutral MSSM Higgs bosons performed
at LEP, Tevatron and LHC~\cite{LEPLimits,TevatronLimits1,CMSLimit,ATLASLimit}. 
In this benchmark scenario the parameters that contributes to 
radiative corrections are fixed such that the mass of the light CP-even Higgs boson, 
$M_h$, is maximal under the variation of $M_A$ and $\tan\beta$. The $m_h^{max}$ scenario allows to set conservative 
lower bounds on $M_A$, $M_H^{\pm}$ and $\tan\beta$~\cite{mhmax2}. However, given the recent discovery of a Higgs
boson with mass $\sim 126$ GeV, this scenario
tend to predict a too high mass for $M_h$, resulting to be, for large part of the MSSM parameter space, 
inconsistent with this observation. This scenario is still currently used in the presented analysis since it offer the possibility to
compare results with past experiments. 

Recently, several benchmark scenarios has been updated~\cite{LHCxsec} to  
accommodate the experimental constraints on past neutral MSSM Higgs searches and the observation of a SM-like Higgs boson.
An interesting updated benchmark scenario is the $m_h^{mod}$ scenario, which has the feature to predict $M_h \simeq 125.5 \pm 3 $ GeV 
for large region of MSSM parameter space.  The $m_h^{mod}$ scenario configuration is obtained by reducing the amount 
of mixing in the stop sector with respect to  the  $m_h^{max}$ scenario. This can be done for both signs of the MSSM parameter that 
regulate the stop mixing $X_t$, giving rise to two complementary scenarios $m_h^{mod+}$ and $m_h^{mod-}$.
The difference between these two scenarios is found to be negligible for experimental searches, the $m_h^{mod+}$ 
benchmark scenario has been used throughout this thesis as reference scenario.

Other interesting benchmark scenario are the light stop scenario and the light stau scenatio.
The first may lead to relevant modification of the gluon fusion production cross section, while the second leads
to modification of the di-photon decay branching fraction of the light CP-even MSSM Higgs boson.
For an overview of other relevant  benchmark scenarios refer to Ref.~\cite{LHCxsec}. 



 


\subsection{Neutral MSSM Higgs Bosons Production and Decay at LHC}
For large region of the MSSM parameter space a SM-like Higgs boson is expected, 
this role is commonly played by the lightest CP-even Higgs boson, $h$. 
Given the Higgs bosons couplings discussed in Section~\ref{sec:couplings} tourns out that the MSSM Higgs bosons $H$ and $A$
tend to be degenerate in mass and decouple from gauge bosons. Furthermore the coupling of the latter
two Higgs bosons with down (up) type fermions are enhanced (suppressed) by $\tan\beta$, therefore, for large $\tan\beta$
bottom-quark and $\tau$ lepton will play an important role for 
the Higgs bosons production and its decays.  

The production of the neutral $CP$-even MSSM Higgs bosons at hadron
colliders proceeds via the same processes as for the SM Higgs
production. The pseudoscalar $A$, instead, cannot be produced
in association with gauge bosons or in vector boson fusion (VBF) processes at
tree-level, as this coupling is forbidden due to $CP$-invariance.  At
the LHC one of the most relevant production mechanisms for the MSSM
Higgs bosons is gluon fusion, $gg\rightarrow A/H/h$. In
addition, the production in association with $b$-quarks becomes
important for large value of $\tan\beta$. These are the only two production mechanism
that are considered in the presented analysis. Figure~\ref{fig:prod} shows the Feynman-diagram
for these processes, while Figure~\ref{fig:xsec} shows the production cross section of the neutral 
MSSM Higgses via these two processes in the $m_h^{max}$ scenario.


The decays of the neutral
MSSM Higgs bosons (in the assumption that all supersymmetric particle
are heavy enough) are the same as for the SM one with the already
cited exception of $A$. Figure~\ref{fig:xsec} shows the decay branching fractions in the $m_h^{mod+}$ scenario
for $h$, $H$ and $A$ as a function of the mass of $A$ for two values of $\tan \beta$. The decay into tau lepton 
pairs is the most important after $b\bar{b}$ and the one used in this thesis. 

\begin{figure}[tp]
     \begin{center}
     \subfigure[]{		
            \includegraphics[height=3.5cm]{feyn_diagrams/diagrams/bbA.pdf}
     }\hspace{0.2cm}	
     \subfigure[]{		
            \includegraphics[height=3.5cm]{feyn_diagrams/diagrams/bbA2.pdf}
     }	\hspace{0.2cm}	
     \subfigure[]{		
            \includegraphics[height=3cm]{feyn_diagrams/diagrams/bbA3.pdf}
     }
     \subfigure[]{	
            \includegraphics[height=3cm]{feyn_diagrams/diagrams/ggH.pdf}
	}	
     \end{center}
    \caption{Feynman diagram for the production of the neutral MSSM Higgs bosons via b-associated (a,b,c) and gluon fusion (d) 
	processes, subsequent decay in tau lepton pairs is considered.}
   \label{fig:prod}
\end{figure}

\begin{figure}[tp]
     \begin{center}

     \subfigure[]{		
            \includegraphics[width=0.6\textwidth]{figure/Xsec/YRHXS_MSSM_neutral_fig6a.pdf}
	}
     \subfigure[]{		
            \includegraphics[width=0.6\textwidth]{figure/Xsec/YRHXS_MSSM_neutral_fig6b.pdf}
	}
    \end{center}
    \caption{Central predictions for the total MSSM Higgs bosons production cross sections via gluon fusion and Higgs radiation off
	bottom quarks  for $\sqrt{s} = 7$ TeV using NNLO and NLO MSTW2008 PDFs $m_h^{max}$ scenario; (a) $\tan\beta  = 5$, (b) $\tan\beta  = 30$.
	Ref.~\cite{LHCxsec1}. }

   \label{fig:xsec}
\end{figure}


\begin{figure}[tp]
     \begin{center}

            \includegraphics[width=0.47\textwidth]{figure/BR_higgs/YRHXS3_BR_fig31.pdf}
            \includegraphics[width=0.47\textwidth]{figure/BR_higgs/YRHXS3_BR_fig32.pdf}
            \includegraphics[width=0.47\textwidth]{figure/BR_higgs/YRHXS3_BR_fig35.pdf}
            \includegraphics[width=0.47\textwidth]{figure/BR_higgs/YRHXS3_BR_fig36.pdf}
            \includegraphics[width=0.47\textwidth]{figure/BR_higgs/YRHXS3_BR_fig37.pdf}
            \includegraphics[width=0.47\textwidth]{figure/BR_higgs/YRHXS3_BR_fig38.pdf}

    \end{center}
    \caption{Branching fraction for the MSSM neutral higgses $h/H/A$ in the $m_h^{mod+}$ scenario for $\tan\beta=10$ and
	$\tan\beta=50$. Ref.~\cite{LHCxsec}.}
   \label{fig:br}

\end{figure}



\subsection{Current Status of the Seacrh for Neutral MSSM Higgs Bosons}

The measure of the couplings of the observed SM-like Higgs boson can shed light on the Higgs sector and determine if this boson
is fully responsable for the generation of all the SM particles masses. 
%In fact, the couplings are sensitive to new physics,
%given the unitarity property of scattering aplitudes for longitudinal vectors and fermions, 
There are two approaches to explore the Higgs sector: one, is to use the measured Higgs couplings with SM particles to 
set constraint on new physics, while the other is to directly search for additional Higgses in a well defined model.

In case the SM-like Higgs boson is interpreted as the light CP-even Higgs boson of the MSSM, the couplings of the Higgs boson 
to vector bosons ($k_V$), up-type fermions ($k_u$) and down-type fermions ($k_d$), can be expressed as a function of  $m_A $ and $\tan\beta$ 
and this allow to set exclusion limits in the $m_A - \tan\beta$ plane~\cite{AtlasConstraint}. Figure~\ref{fig:ex1} shows the exclusion limits in a 
``simplified MSSM'' model~\cite{sympleMSSM1,sympleMSSM2} via fits to the measured rates of Higgs boson production and decay.

 
\begin{figure}[tp]
     \begin{center}

            \includegraphics[width=0.8\textwidth]{figure/limits/constraintAtlas.pdf}

    \end{center}
    \caption{Regions of the  $m_A - \tan\beta$ plane excluded in a simplified MSSM model via fits to the measured
%rates of Higgs boson production and decays. The likelihood contours where $−2 \ln \Lambda = 6.0$, corresponding
approximately to 95\% CL ($2\sigma$), are indicated for the data and expectation assuming the SM Higgs sector.
The light shaded and hashed regions indicate the observed and expected exclusions, respectively. The
SM decoupling limit is $m_A \rightarrow \infty$. See Reference~\cite{AtlasConstraint}.}

   \label{fig:ex1}
\end{figure}


The current latest constraint on $m_A - \tan\beta$  by direct search of neutral MSSM Higgs bosons~\cite{} are instead shown in Figure~\ref{fig:ex2}
and are part of the work of this thesis.



 
\begin{figure}[tp]
     \begin{center}

            \includegraphics[width=0.8\textwidth]{figure/blank.pdf}

    \end{center}
    \caption{Limit CMS or ATLAS depending if we manage to publish in time}


   \label{fig:ex2}
\end{figure}




















