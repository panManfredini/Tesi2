\chapter{Object Reconstruction, Preselection and Efficiency Corrections}
\label{sec:presel}

In this section the preselection and reconstruction criteria for the
objects used in this analysis are presented.  For each object and
selection criteria all corrections that have been applied to data and
MC are also described.  A summary of the preselection on physics
objects used in this analysis is reported in Table~\ref{tab:presel}.


\section{Electrons}
\label{sec:presel:elec}

This analysis uses electrons found by the standard electron
identification algorithms ~\cite{AtlasCSCBook} that pass the {\tt Medium++}
criteria. A preselection is applied to the electrons to ensure that
the electron cluster has a transverse energy of $E_{T} > 15 GeV$, is
within the pseudorapidity range $|\eta|<2.47$, but is outside of the region
$1.37<|\eta|<1.52$. The first requirement ensures that the selected
electrons are within a range of $E_{T}$ where the electron reconstruction
and trigger efficiencies are well understood. The further requirements
ensure that the electron is reconstructed within the acceptance of
the ATLAS tracking, but outside of the transition region between the
barrel and end-cap calorimeters. 
In addition, the author is required to be either one or three, to ensure that the electron was 
reconstructed with either the standard electron algorithm or both the
standard and soft electron algorithms, respectively.
Finally, to ensure that the electron is not reconstructed within a region of the
calorimeter with readout problems, dead or non-nominal high voltage
conditions or suffering from high noise, the electron is rejected if
the cluster $\eta$ and $\phi$ position match a flagged region in the
Object Quality maps (OQ maps)~\cite{WZcross} provided by the egamma
Performance group.

For the electrons used in this analysis, the four-vector of the
particle is defined using the energy of the electron calorimeter
cluster and the direction of the electron track. Selections that
involve the electron position in the calorimeter, in this analysis the
$\eta$ and the OQ map selections, are made using a four-vector built
entirely from the electron cluster properties. Both the energy scale
and resolution of the electrons used in this analysis are corrected,
following the recommendations of the EGamma performance group, by
using the {\tt egammaAnalysiUtils}
package\cite{EGammaRecomendations}. Energy scale corrections are
applied to electrons in data, whereas an additional smearing is
applied to the electron energy in MC.

In addition to the preselection defined above, isolation criteria are
defined to select electrons with little or no activity around
them. The calorimetric isolation, \etcone, is calculated as the sum of
the transverse energy of the additional topological clusters in the
electromagnetic and hadronic calorimeters in a cone of $\Delta R <
0.2$ around an electron\footnote{The $\Delta R$ variable is defined by
$\Delta R=\sqrt{(\Delta\eta)^2+(\Delta\phi)^2}$, where $\Delta \eta$
and $\Delta \phi$ correspond to the difference between
the pseudorapidities and azimuthal angles of the objects considered, respectively.}
. The summed transverse energy is corrected, as a function of the
number of primary vertices in the event, to reduce the dependence on
pileup. In addition, the track isolation, \ptcone, is defined as the
scalar sum of the $\pt$ of all additional tracks with $\pt>1$ GeV in a
cone of radius $\Delta R<0.4$ around an electron. In this analysis, an
electron with $\etcone/\pt<0.08$ and $\ptcone/\pt<0.06$ is considered isolated.


\section{Muons}
\label{sec:presel:muon}

Muons reconstructed by the STACO algorithm \cite{AtlasCSCBook} are
used in this analysis - those passing the STACO Loose quality criteria
are considered at the preselection stage, whereas the more stringent
STACO Combined quality criteria are required for the final muon
selection. Muons with a transverse momentum $\pt > 10$ GeV and
within the pseudorapidity range $|\eta| < 2.5$ are selected. The
difference between the $z$ position of the muon track extrapolated to
the beam line and the primary vertex $z$ position must be less than 10
mm. 

Further quality criteria are placed on the Inner Detector track of
the muon candidate to ensure that it is well reconstructed and to
reduce the fake rate due to decays of hadrons in flight. These
requirements ensure that multiple hits are found on the track in the
various layers of the ID, but take into account that dead or
uninstrumented regions may be crossed by the muon. Firstly, if the
muon passes through a section of the b layer of the Pixel detector
that is instrumented and not suffering from detector problems, there
should be one or more b layer hits on the track. The sum of the number
of hits on the track in the Pixel detector and the number of crossed
dead Pixel detector layers should be at least one. The sum of the
number of hits within the SCT detector and the number of dead SCT
modules crossed should be five or greater. The total number of crossed
dead Pixel detector and SCT detector layers should be less than three.
When within the angular region $|\eta|<1.9$, the sum of the TRT hits
and outliers on the track must be greater than five and the ratio of
TRT outlier hits to the total number of TRT hits must be less than
0.9. When the muon track is in the region $|\eta|\ge1.9$, the ratio of
TRT outlier hits to the total number of TRT hits must be less than 0.9
only if the sum of the TRT hits and outliers on the track is be
greater than five.

The momentum scale and resolution of the muons in this analysis are
corrected in MC following the recommendations of the Muon Combined
Performance group. The momentum corrections were measured by comparing
the di-muon mass peak position and resolution between data and MC at
the Z resonance. Smearings are applied in a coherent manner to
the ID, MS extrapolated and combined momenta of the transverse
momentum of the muon. In addition, a scale correction is applied to
the combined momentum momentum.

As for the electrons used in this analysis, both calorimeteric and
track based isolation are used to require little or no activity around them,
in addition to the preselection above. The muon \etcone and \ptcone
variables are defined as for the electron case and are calculated
in cones of $\Delta R<0.2$ and $\Delta R<0.4$ around the muon, respectively. 
Once more \etcone is corrected as a function of the number of primary vertices in the
event, to reduce the dependence on pileup. In this analysis, a muon with
$\etcone/\pt<0.04$ and $\ptcone/\pt<0.06$ is considered isolated.


\section{Jets}
\label{sec:presel:jet}

The jets used in this analysis are reconstructed using the Anti-$k_T$
algorithm \cite{AntiKT} with the distance parameter R=0.4 taking
topological clusters as inputs. The reconstructed jets are calibrated
to the Local Cluster Weighting (LCW) scale \cite{jetenergyscale}. In
addition, the effect of pileup on the reconstructed energy is reduced
by applying a further correction based on the pile-up area method with
a final in-situ calibration also applied.

A preselection is then applied that requires the reconstructed jets
have a transverse momentum, after calibration, of $\pt > 30$ GeV and
to be within the pseudorapidity range $|\eta| < 4.5$. The effect of pileup on
the reconstructed jets is further reduced by requiring that jets with
the pseudorapidity range $|\eta|<2.4$ and a transverse momentum of $\pt <
50$ GeV have a absolute value of the Jet Vertex Fraction (JVF) of greater than 0.5.

A separate set of preselected jets is defined that are used only for
b-tagging (henceforth known as ``taggable jets''). Such jets are
reconstructed and calibrated as for the standard preselected
jets. However, the taggable jets are required to have a transverse
momentum of $\pt > 20$ GeV and to have a reconstructed pseudorapidity
of $|\eta| < 2.5$. The second requirement ensures that charged
particles within the jets pass through the tracking volume and hence
can be used for b-tagging of the jet. Finally, the same JVF selection
as the standard preselected jets is applied to the taggable jets.

\subsection{b-Tagging}
\label{sec:presel:btag}

The tagging of jets due to the hadronisation of b-quarks is performed
using the MV1 b-tagging algorithm \cite{mv1}. This neural network based
algorithm uses the output weights of the JetFitter+IP3D, IP3D and
SV1 b-taggers as inputs. The working point that gives a nominal
b-tagging efficiency of 70\% on \ttbar samples is used.

\section{Taus}
\label{sec:presel:tau}

Hadronically decaying tau candidates are reconstructed using clusters
in both the electromagnetic and hadronic calorimeters. A preselection
is applied to the candidates that requires the reconstructed $\tau$
candidates to have a transverse momentum of $\pt>20$ GeV and to have
a reconstructed pseudorapidity of $|\eta| < 2.5$. Furthermore, it is
required that the candidates have either one or three tracks within a
cone of $\Delta R < 0.2$ associated to them and have a charge of $\pm
1$. Finally, the preselected tau candidates should pass the BDT-Medium
multivariant tau identification selection as well as the dedicated
electron and muon vetoes for hadronically decaying tau candidates.


\section{Overlap Removal}
\label{sec:presel:olr}

After the preselection of the physics objects needed for this
analysis, an overlap removal between the different objects is then
applied to avoid double-counting.  The distance between two objects in
rapidity $\Delta\eta$ and polar angle $\Delta\phi$ is defined as
$\Delta R=\sqrt{(\Delta\eta)^2+(\Delta\phi)^2}$. Overlap removal is
then applied in the following order:

\begin{itemize}
\item preselected electrons are removed if they overlap with a preselected muon  within $\Delta R < 0.2$,
\item preselected taus are removed if they overlap with a preselected muon or electron within $\Delta R < 0.2$,
\item preselected jets are removed  if they overlap with a preselected
  muon, electron or tau within \linebreak $\Delta R<0.2$.
\end{itemize}

\section{Missing Transverse Energy}
\label{sec:presel:met}

The missing transverse energy, \MET, is calculated using the
RefFinal method, which takes the energy
deposited in the calorimeter, the muons reconstructed in the muon
spectrometer and tracks reconstructed in the inner detector as inputs. 
For this, the energy deposits are calibrated based
upon the high-\pt physics object they are associated to, with an order
of preference of electrons, photons, hadronically decaying taus, jets
and finally muons. Any unassociated energy deposits are combined into
the so-called ``soft-term''. To reduce the effect of pileup on the
\met calculation, corrections are applied to both the jets in an event
and to the soft-term. Firstly, any jet with a pseudorapidity of $|\eta|<2.4$
that enters the \met calculation is weighted by it's JVF. Similarly,
the soft-term is weighted by the soft-term-vertex-fraction (STVF) of
the event - the ratio given by
\begin{equation}
STVF = \frac{\sum_{track, PV}\pt}{\sum_{track}\pt}
\end{equation}
where $\sum_{track, PV}\pt$ is the sum of the transverse momentum of
all tracks associated to the primary vertex, but unmatched to physics
objects, and $\sum_{track}\pt$ is the sum of
the transverse momentum of all tracks in the event unmatched to
physics objects. Any calibration applied
to the energy or direction of the physics objects in the final
analysis is also propagated to the \met.


\section{Vertices} 
\label{sec:presel:vertices}
In this analysis vertices are selected that have a minimum of three
associated tracks: this helps to ensure that the selected vertices
come from beam-beam interactions rather than, for instance, cosmic
muons.

\section{Event Cleaning}
\label{sec:presel:celaning}

In addition to the data quality requirements described in section
\ref{sec:samples:dq}, further selections are applied to veto events
where bad jets are identified as arising from detector effects
(coherent noise in the EM and Tile calorimeters or spikes in the HEC
calorimeter), cosmics or beam based background. To reject events, the
recommendations of the JetEtMiss performance group \cite{jetMETreco}
are followed: Events are rejected if at least one AntiKt4LCTopo jet
with $p_{T}>20$ GeV, that passes the overlap removal with electrons,
muons and taus described in section \ref{sec:presel:olr}, fails the {\tt
BadLooseMinus} selection or points towards the hot Tile Calorimeter
cells identified in data taking periods B1 and B2 \cite{hottile}.


\begin{table}
  \begin{center}
    \begin{tabular}{cc}
      \hline \hline
      Physics Object & Preselection \\

      \hline
      Electrons & $\pt >15 $ GeV \\
      & $|\eta|<1.37$ or $1.52<|\eta|<2.47$ \\
      & {\tt Medium++} \\
      & Author = 1 or 3 \\
      & Pass Object Quality Flag \\
      \hline
      Muons & $\pt>10 $ GeV  \\	
      & $|\eta|<2.5$ \\
      & isLoose STACO muon \\
      & Inner Detector track quality requirements \\
      & Inner Detector track $|z_0^{PV}|<10\mathrm{mm}$ \\
      \hline
      Jets & $\pt>30$ GeV  \\
      & $|\eta|<4.5$ \\
      & $|JVF|>0.5$ for jets with $|\eta|<2.4$ and $\pt<50$ GeV \\
      \hline
      Jets (taggable) & $\pt>20$ GeV \\
      & $|\eta|<2.5$ \\
      & $|JVF|>0.5$ for jets with $|\eta|<2.4$ and $\pt<50$ GeV \\
      \hline
      Taus & $\pt>20$ GeV \\	
      & $|\eta| < 2.5$ \\
      & BDT Medium \\
      & $N_{tracks} = 1 \mathrm{or} 3$ \\
      & Author = 1 or 3 \\
      & Muon and Electron Veto \\
      \hline
      \MET & RefFinal with STVF correction\\	
      \hline
      Vertices & $N_{tracks} \ge 3$ \\	
      \hline \hline
    \end{tabular}
    \caption{Summary of the preselections used for physics objects in this analysis}
    \label{tab:presel}
  \end{center}
\end{table}


\section{Monte Carlo Corrections}
\label{sec:presel:MCCorr}

The MC samples used on this analysis are corrected to account for differences
between the simulation and data in the trigger, lepton reconstruction and
identification and b-tagging efficiencies. Furthermore, the MC is
reweighted so that the vertex multiplicity distribution agrees with
that in the data.

\section{Trigger Efficiency corrections}
\label{sec:presel:TriggerCorr}

Correction factors are applied to the simulated trigger efficiency for
both the single electron, \linebreak \verb=EF_e24vhi_medium1=, and combined
electron-muon, \verb=EF_e12Tvh_medium1_mu8=, triggers used in this
analysis. The trigger efficiency for the \verb=EF_e24vhi_medium1= has
been measured with respect to offline electrons using a tag and probe
method in \Zee events \cite{SingleElecTrigSF}. Scale factors are derived from the ratio of the
trigger efficiency measured in data and MC, measured as a
function of electron \pt and $\eta$.

For the \verb=EF_e12Tvh_medium1_mu8= trigger, correction factors are
measured separately for the two individual legs of the trigger,
\verb=EF_e12Tvh_medium1= and \verb=EF_mu8= \cite{SingleEMuTrigSF}. The product of the two
correction factors is then used as the overall scaling factor. The
trigger efficiency for the \verb=EF_e12Tvh_medium1= leg has been
measured with respect to offline electrons using a tag and probe
method for \Zee events in both data and MC. Likewise, the
\verb=EF_mu8= trigger efficiency scale factors are derived using a tag
and probe measurement with \Zmumu events. Oncemore, scale factors are derived
from the ratio of the trigger efficiency measured in data and MC, measured as a
function of electron \pt and $\eta$.

\section{Lepton Reconstruction Efficiency Corrections}
\label{sec:presel:LepEffCorr}

Further correction factors are applied to the MC samples to account
for differences in the lepton reconstruction and identification
efficiencies between data and simulation. Scale factors for the electron
identification and reconstruction efficiencies are measured separately
using a combination $\Zee$ and $J / \psi \rightarrow e e$ tag and probe
measurements \cite{ElecSF}. Both sets of scale factors are measured as a function of the electron $E_{T}$ and $\eta$. 

Similarly, muon reconstruction efficiency scale factors
have been measured, using a \Zmumu tag and probe analysis, 
as a function of the muon \pt, $\eta$, $\phi$ and charge \cite{MuonSF}.

\section{b-tagging Efficiency Corrections}
\label{sec:presel:BTagEffCorr}

Corrections are applied to the b-tagging efficiency and mistag rate in
MC, using a combination of the System8 and likelihood scale
factor measurements \cite{BTagSF}. Separate scale factors are applied based on the origin of the jet at truth level - from a $b$ quark, a $c$ quark, a $\tau$ or a light quark - and are applied as a function of the jet \pt and $\eta$.

\section{Pileup Reweighting}
\label{sec:presel:PileupCorr}

Differences between the distribution of the average number of
interactions per bunch crossing, $<\mu>$, in MC and data are
corrected by reweighting the MC $<\mu>$ distribution to that in the
full considered dataset. An additional scaling of $1.1 \times <\mu>$
is applied to the MC, which has been shown to improve the description
of the number of primary vertices distribution of the data.
